\documentclass[main.tex]{subfiles}

\begin{document}
\section*{Grundbegriffe}
\subsection*{Mengen}
\begin{karte}{Definition Menge}
    Eine Zusammenfassung gewisser unterscheidbarer Objekte, 
    der sogenannten Elemente dieser Menge.
\end{karte}
\begin{karte}{Definition Mächtigkeit}
    Die Anzahl der Elemente der Menge. Ist die Menge unendlich, 
    so ist die Mächtigkeit \( \infty \).
\end{karte}
\begin{karte}{Produkt von Mengen}
    \[ A \times B := \set{ (a,b) \;\vert \; 
    a \in A, b \in B }. \]
\end{karte}
\subsection*{Relationen}
\begin{karte}{Definition Relation}
    Eine Relation auf einer Menge \(X\)  ist eine Teilmenge 
    von \( X \times X \).
\end{karte}
\begin{karte}{Definition Äquivalenzrelation}
    Eine Äquivalenzrelation ist:
    \begin{description}
        \item[reflexiv] \( \forall x \in X : x \sim x \).
        \item[symmetrisch] \( x \sim y \Rightarrow y \sim x \).
        \item[transitiv] \( x \sim y \wedge y \sim z \Rightarrow 
        x \sim z \).
    \end{description}
\end{karte}
\begin{karte}{Definition Äquivalenzklasse}
    \[ [ x ] := \set{ y \in X \;\vert \; y \sim x }. \]
    Ein Element einer Äquivalenzklasse \( [x] \) heißt 
    Vertreter oder Repräsentant von \( [x] \).
\end{karte}
\begin{karte}{Definition Quotientenmenge}
    Die Menge aller Äquivalenzklassen 
    \[ X / \sim := \set{ [x] \;\vert \; x \in X } \]
    ist die Quotientenmenge von \( X \) nach \( \sim \).
\end{karte}
\begin{karte}{Definition Kongruenz modulo \(n\)}
    Die Äquivalenzrelation \( \equiv_n \) auf \( \Z \) beschreibt die 
    Kongruenz modulo \(n\):
    \[ x \equiv_n y \Leftrightarrow n \;\vert \; (x-y) 
    \ (\text{\glqq{}} n \text{ teilt } x-y \text{\grqq{}}). \] 
\end{karte}
\begin{karte}{Quotientenmenge Kongruenz modulo \(n\)}
    Die Quotientenmenge wird mit \( \Z / n\Z \) bezeichnet. \\
    Sie enthält genau \(n\) Äquivalenzklassen, nämlich 
    \( [0], \ldots, [n-1] \).
\end{karte}
\begin{karte}{Definition teilweise Ordnung}
    Eine teilweise Ordnung \( \prec \) ist:
    \begin{description}
        \item[reflexiv] \( \forall x \in X : x \prec x \).
        \item[antisymmetrisch] \( x \prec y \wedge y \prec x 
        \Rightarrow y = x \).
        \item[transitiv] \( x \prec y \wedge y \prec z \Rightarrow 
        x \prec z \).
    \end{description}
    Eine Menge \(X\) mit der teilweisen Ordnung \( \prec \) nennt 
    man auch teilweise geordnete Menge.\\
    Gilt für je zwei Elemente \( x,y \) der Menge \( x \prec y \) 
    oder \( y \prec x \), so ist \(X\) streng geordnet.
\end{karte}
\begin{karte}{Definition obere Schranke}
    Ein Element \( s \in X \) ist eine obere Schranke für 
    \( T \subset X \), wenn \( z \prec s \,\forall z \in T \) 
    gilt.\\
    Ein Element \( m \in X \) ist maximal, wenn aus 
    \( m \prec z \) stets \( m = z \) folgt.
\end{karte}
\subsection*{Abbildungen}
\begin{karte}{Definition Abbildung}
    Eine Abbildung \(f : X \rightarrow Y\) ist eine Vorschrift, 
    durch die jedem \(x \in X \) genau ein Element \(f(x) \in Y\)
    zugeordnet wird:
    \[ x \mapsto f(x)\] \\
    \( f(x) \) ist das Bild von \(x\). \\
    Der Definitionsbereich von \(f\) ist \(X\). \\
    Der Zielbereich von \(f\) ist \(Y\). 
\end{karte}
\begin{karte}{Definition Bild und Urbild}
    \( f : X \rightarrow Y \) ist eine Abbildung. \\
    Ist \(A \subset X\) und \(B \subset Y\), 
    dann ist das Bild von \(A\):
    \[ f(A) := \set{ f(x) \;\vert \; x \in A }. \]
    Auch ist \( \Bild(f) := f(X)\). \\
    Das Urbild von \(B\) ist: 
    \[ \inverse{f}(B) := \set{x \;\vert \; f(x) \in B}. \]
\end{karte}
\begin{karte}{Definition Injektivität}
    Die Abbildung \(f: X \rightarrow Y\) ist injektiv, \\
    wenn keine zwei Elemente von \(X\) auf dasselbe 
    Element von \(Y\) abgebildet wird. 
\end{karte}
\begin{karte}{Definition Surjektivität}
    Die Abbildung \(f: X \rightarrow Y\) ist surjektiv, 
    wenn
    \[ f(X) = Y. \]
\end{karte} 
\begin{karte}{Definition Bijektivität}
    Ist eine Abbildung injektiv und surjektiv, 
    dann ist sie bijektiv. 
\end{karte}
\begin{karte}{Definition Identität}
    Die Abbildung \( \id_A : A \rightarrow A, 
    \id_A(a) = a\) ist die Identität.
\end{karte}
\begin{karte}{Defintion kanonische Abbildung}
    Ist \( \sim \) eine Äquivalenzrelation auf \(X\). \\
    Die surjektive Abbildung \(X \rightarrow X/\sim, x \mapsto [x]\), 
    ist die kanonische Projektion oder Quotientenabbildung. 
\end{karte}
\begin{karte}{Definition Wohldefiniertheit}
    Eine Abbildung \( f: X \rightarrow Y, 
    x \mapsto f(x) \) ist wohldefiniert, 
    wenn gilt:
    \[ \forall x \in X \, \existse y \in Y: f(x) = y \]
\end{karte}
\begin{karte}{Definition Komposition}
    Für \( f: X \rightarrow Y \) und 
    \( g: Y \rightarrow Z \) ist die Komposition:
    \[ g \circ f : X \rightarrow Z, x \mapsto g(f(x)) \]
\end{karte}
\begin{karte}{Definition Kommutatives Diagramm}
    Wenn in einem Diagramm zu je zwei Mengen alle Abbildungen 
    (auch Kompositionen), die die eine Menge in die andere abbilden,
    übereinstimmen, dann nennt man das Diagramm kommutativ. 
\end{karte}
\begin{karte}{Definition Inverses}
    Für eine bijektive Abbildung \( f: X \rightarrow Y \) ist
    \[ \inverse{f}: Y \rightarrow X, f(x) \mapsto x \] 
    die Umkehrabbildung oder Inverses.
\end{karte}
\begin{karte}{Definition Einschränkung}
    Eine Einschränkung von \(f: X \rightarrow Y\) 
    auf \(A \subset X\) ist:
    \[f \vert_A : A \rightarrow Y, x \mapsto f(x)\]
\end{karte}
\subsection*{Körper}
\begin{karte}{Definition Innere Verknüpfung}
    Eine Abbildung \(f: M \times M \rightarrow M\) 
    ist eine innere Verknüpfung auf der Menge \(M\).
\end{karte}
\begin{karte}{Defintion Gruppe}
    Eine Gruppe \( (G,*) \) besteht aus einer Menge 
    \(G\) und einer inneren Verknüpfung \(*\) auf \(G\).\\
    Eine Gruppe hat folgende Eigenschaften: 
    \begin{description}
        \item[Assoziativität] 
        \( \forall a,b,c \in G : 
        (a * b) * c = a * (b * c) \).
        \item[Existenz des neutralen Elements] 
        \( \existse \, e \in G :\forall a \in G : 
        e*a = a*e = a \).
        \item[Existenz des inversen Elements] 
        \( \forall a \in G \, \existse b \in G : 
        a * b = b * a = e \). 
    \end{description}
    Gilt insbesondere \(\forall a,b \in G : 
    a * b = b * a \), so ist \( (G,*) \) abelsch 
    oder kommutativ. \\
    Beispiele für Gruppen : 
    \begin{itemize}
        \item \( (\Q, +) \)
        \item \( (\Z, +) \)
        \item \( (\N_0, +) \) ist keine Gruppe, da keine 
        Inversen existieren.
        \item \( (\set{ x\in\R \;\vert \; x > 0 }, \cdot) \) 
    \end{itemize}
\end{karte}
\begin{karte}{Definition Körper}
    Ein Körper \( (K,+,\cdot) \) besteht aus einer Menge 
    \(K\) und \\ 
    zwei inneren Verknüpfungen. Ein Körper hat folgende Eigenschaften: 
    \begin{description}
        \item[Abelsche Gruppe der Addition] 
        \( (K,+) \) ist eine abelsche Gruppe.
        \item[Abelsche Gruppe der Multiplikation]
        \( (K^*,\cdot) \) ist eine abelsche Gruppe\\
        mit \(K^* = K \setminus \set{0}\).
        \item[Distributivität]
        \( \forall a,b,c \in K : 
        a \cdot (b + c) = a \cdot b + a \cdot c\).
    \end{description}
    Beispiele für Körper: 
    \begin{itemize}
        \item \( (\Q, +, \cdot) \)
        \item \( (\R, +, \cdot) \)
        \item \( (\Z, +, \cdot) \) ist kein Körper, da keine 
        multiplikativ Inversen existieren.
        \item \( (\C, +, \cdot) \)
        \item \( (\F_p, +, \cdot) \), \(p\) Primzahl
    \end{itemize}
\end{karte}
\begin{karte}{Null- und Einselement im Körper}
    Das neutrale Element der Addition wird Nullelement genannt und 
    das neutrale Element der Multiplikation wird Einselement genannt.
\end{karte}
\begin{karte}{Definition Teilkörper}
    Für \( L \subset K \) und \( (K,+,\cdot) \) ist \(L\) 
    ein Teilkörper, wenn sich die Addition \(+\) und die 
    Multiplikation \( \cdot \) zu inneren Verknüpfungen 
    auf \(L\) einschränken, sodass \(L\) selbst zu einem 
    Körper wird.
\end{karte}
\begin{karte}{Der Körper der komplexen Zahlen}
    \[ \C = \set{ z \;\vert \; z = x + i \cdot y, \; x, 
    y \in \R, i^2 = -1 } \]
    \( x \) ist der Realteil einer komplexen Zahl \(z\) und 
    \( y \) ist der Imaginärteil der komplexen Zahl.\\
    Für die Addition gilt:
    \[ (u,v) + (x,y) := (u + x, v + y). \]
    Für die Multiplikation gilt: 
    \[ (u,v) \cdot (x,y) := (ux-vy, ux+vy). \]
    Für das multiplikativ Inverse gilt:
    \[ \inverse{(x,y)} = \left( \frac{x}{x^2 + y^2}, 
    -\frac{y}{ x^2 + y^2 } \right) \]
\end{karte}
\begin{karte}{Definition Norm und komplexe Konjugation}
    Das Norm einer Zahl \( z=x+iy \in \C \) lautet 
    \( \abs{z} = \sqrt{x^2+y^2} \) und das komplex 
    Konjugierte von \(z\) lautet \( \bar{z} = x - iy \).
\end{karte}
\end{document}
