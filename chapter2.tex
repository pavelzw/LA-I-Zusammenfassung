\documentclass[main.tex]{subfiles}

\begin{document}

\section*{Vektorräume}
\subsection*{Der Begriff des Vektorraums}
\begin{karte}{Definition Vektorraum}
    Ein \( K \)-Vektorraum ist eine Menge \(V\) mit der inneren 
    Verknüpfung Vektoraddition 
    \[ V \times V \rightarrow V, (v,w) \mapsto v + w \]
    und der skalaren Multiplikation 
    \[ K \times V \rightarrow V, (\alpha, v) \mapsto 
    \alpha \cdot v. \]
    so dass gilt: 
    \begin{description}
        \item[Abelsche Gruppe der Addition]
        \( (K,+) \) ist eine abelsche Gruppe.  
        \item[Distributivität von Addition und skalarer Multiplikation] \hfill \\ 
        \( \forall \alpha, \beta \in K\) und \( v, w \in V\) gelten : \\
        \( (\alpha + \beta) \cdot v =  \alpha \cdot v + \beta \cdot v\)
        und 
        \( \alpha \cdot ( v + w) = \alpha \cdot v + \alpha \cdot w\)
        \item[Assoziativität der skalaren Multiplikation] \hfill \\
        \( \forall \alpha, \beta \in K\) und \( v \in V\) gelten :
        \((\alpha \beta) \cdot v = \alpha \cdot (\beta \cdot v)\)
        \item[Existenz eines neutralen Elements]
        \( \forall v \in V : \exists 1_K \cdot v = v\)
    \end{description}
\end{karte}
\begin{karte}{Definition Untervektorraum}
    Ein Untervektorraum \(U\) vom \(K\)-Vektorraum \(V\) ist eine nicht-leere
    Teilmenge \(U \subset V\), für die gilt : 
    \[\alpha \in K \text {und } u, v \in U \Rightarrow \alpha \cdot v \in U
    \text{ und } u + v \in U\]
    Ein \(K\)-Untervektorraum ist ein \(K\)-Vektorraum 
    (bez. der eingeschränkten Addition und skalaren Multiplikation).
\end{karte}
\begin{karte}{Definition affiner Unterraum}
    Ist \(A\) ein affiner Unterraum \(A \subset V\) eines Vektorraums \(V\),
    so existieren ein Untervektorraum \(U \in V\) und ein Vektor \(v_0 \in V\) 
    und es gilt: 
    \[A = v_0 + U := \set{v_0 + v \;\vert\; v \in U}\] 
    Ist \(U = \set{\lambda \cdot w \;\vert\; \lambda \in K}\) 
    mit \(w \in V \backslash \set{0}\) heißt \(A\) affine Gerade. \\
    Ein affiner Untervektorraum \(A \subset V\) ist genau dann ein Untervektorraum,
    wenn \(0 \in A\).
\end{karte}
\end{document}