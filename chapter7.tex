\documentclass[main.tex]{subfiles}

\begin{document}
\section*{Klassifikation von Endomorphismen}
\subsection*{Eigenwerte und Eigenvektoren}

\begin{karte}{Definition Eigenwert, Eigenvektor, Eigenraum\\von Endomorphismen}
    Sei \( f: V \rightarrow V \) ein Endomorphismus. Unter 
    einem \textit{Eigenvektor} von \(f\) zum \textit{Eigenwert} 
    \(\lambda \in K\) versteht man einen Vektor \(v \neq 0\) aus 
    \(V\) mit 
    \[ f(v) = \lambda v. \]
    Der \textit{Eigenraum} zu einem Eigenwert \( \lambda \) von 
    \(f\) ist definiert als 
    \[ E_\lambda(f) = \set{ v \in V \;\vert \; f(v) = \lambda v } 
    = \ker(f - \lambda \cdot \id_V). \]
\end{karte}
\begin{karte}{Definition Eigenwert, Eigenvektor, Eigenraum\\von Matrizen}
    Eigenvektoren, Eigenwerte und Eigenräume von 
    \( L(A) : K^n \rightarrow K^n, A \in M_n(K) \), bezeichnet 
    man auch als \textit{Eigenvektoren}, \textit{Eigenwerte}, 
    \textit{Eigenräume} von \(A\).
\end{karte}
\begin{karte}{Berechnung von Eigenwerten\\ohne charakteristisches Polynom}
    Sei \( f: V \rightarrow V \) ein Endomorphismus und 
    \( \lambda \in K \). Dann ist \( \lambda \) genau dann ein 
    Eigenwert von \( V \), wenn 
    \( \ker(f - \lambda \id_V) \neq 0 \). Ist 
    \( \dim_K(V) < \infty \), so ist diese Aussage äquivalent zu 
    \[ \det(f - \lambda \id_V) = 0. \]
\end{karte}
\begin{karte}{Definition Spektrum}
    Sei \(V\) ein \(K\)-Vektorraum. Sei 
    \( f: V \rightarrow V \) ein Endomorphismus. Dann 
    bezeichnen wir 
    \[ \spec_K f := \set{ \lambda \in K 
    \;\vert \; f - \lambda \id_V \text{ nicht invertierbar} } \]
    als das \textit{Spektrum} von \(f\) über \(K\). Ist 
    \( A \in M_n(K) \), so setzen wir 
    \[ \spec_K A := \spec_K(L(A)). \]
\end{karte}
\begin{karte}{Eigenvektoren von Eigenwerten}
    Sei \( \abb{f}{V}{V} \) ein Endomorphismus. Seien 
    \( \lambda_1,\ldots,\lambda_n \) verschiedene Eigenwerte 
    von \(f\). Sind \( v_1, \ldots, v_n \) Eigenvektoren von 
    \(f\) zu den Eigenwerten \( \lambda_1, \ldots, \lambda_n \), 
    so ist \( (v_1,\ldots,v_n) \) linear unabhängig.
\end{karte}
\begin{karte}{Direkte Summe der Eigenräume}
    Ist \( \abb{f}{V}{V} \) ein Endomorphismus und sind 
    \( \lambda_1,\ldots,\lambda_n \) verschiedene Eigenwerte 
    von \(f\), so ist die Summe der Untervektorräume 
    \( E_{\lambda_1}(f) + \cdots + E_{\lambda_n}(f) \) eine 
    direkte Summe.
\end{karte}
\begin{karte}{Maximalanzahl Eigenwerte}
    Ist \( \abb{f}{V}{V} \) ein Endomorphismus eines 
    \(n\)-dimensionalen Vektorraums, so besitzt \(f\) 
    höchstens \(n\) verschiedene Eigenwerte.
\end{karte}
\subsection*{Diagonalisierbarkeit}
\begin{karte}{Definition Diagonalisierbarkeit}
    Ein Endomorphismus, für den eine Basis von Eigenvektoren 
    existiert, heißt \textit{diagonalisierbar}. Eine Matrix 
    \( A \in M_n(K) \) heißt \textit{diagonalisierbar}, falls 
    \( L(A) \) diagonalisierbar ist.\\
    Ein Endomorphismus eines \(n\)-dimensionalen Vektorraums, 
    der \(n\) verschiedene Eigenwerte hat, ist diagonalisierbar. 
    Für quadratische Matrizen gilt die analoge Aussage.\\
    Eine Matrix \( A \in M_n(K) \) ist genau dann diagonalisierbar, 
    wenn sie einer Diagonalmatrix ähnlich ist.
\end{karte}
\begin{karte}{Definition Ähnlichkeit}
    Wir nennen zwei Matrizen \( A, B \in M_n(K) \) 
    \textit{ähnlich} oder \textit{zueinander konjugiert}, 
    wenn es ein \( S \in \GL_n(K) \) gibt mit 
    \[ A = \inverse{S} \cdot B \cdot S. \]
    Die durch Ähnlichkeit definierte Relation \( \sim \) 
    auf \( M_n(K) \) ist eine Äquivalenzrelation.\\
    Ähnliche Matrizen haben das gleiche Spektrum.\\
    Ähnliche Matrizen haben das geiche charakteristische Polynom.\\
    Ähnliche Matrizen haben die gleiche Spur.
\end{karte}
\begin{karte}{Definition Diagonalmatrix}
    Eine Matrix \( A = (a_{ij}) \in M_n(K) \) heißt 
    \textit{Diagonalmatrix}, \\
    wenn \( a_{ij} = 0  
    \; \forall i,j \in \set{1,\ldots,n} \) mit \( i \neq j \).
\end{karte}
\begin{karte}{Definition geometrische Vielfachheit}
    Sei \( \lambda \in K \) ein Eigenwert eines Endomorphismus 
    \( f \). Die \textit{geometrische Vielfachheit} von 
    \( \lambda \) ist die Dimension von \( E_\lambda(f) \).\\
    Ein Endomorphismus eines \(n\)-dimensionalen Vektorraums 
    ist genau dann diagonalisierbar, wenn die Summe der 
    geometrischen Vielfachheiten aller Eigenwerte gleich 
    \(n\) ist.
\end{karte}
\subsection*{Das charakteristische Polynom} %Polynomringe dazu
\begin{karte}{Definition Ring}
    Ein \textit{Ring} ist eine Menge \(R\) mit zwei inneren 
    Verknüpfungen \(+\) (Addition) und \( \cdot \) 
    (Multiplikation) derart, dass 
    \begin{enumerate}
        \item \( (R, +) \) eine abelsche Gruppe ist.
        \item \( \cdot \) assoziativ ist und ein neutrales 
        Element besitzt.
        \item \( \forall x,y,z \in R \) die Distributivgesetze 
        \[ x \cdot (y + z) = x \cdot y + x \cdot z \]
        \[ (y + z) \cdot x = y \cdot x + z \cdot x \]
        gelten.\\ 
        Ist darüber hinaus die Multiplikation kommutativ, nennt man 
        \(R\) einen \textit{kommutativen Ring}. Die neutralen Elemente 
        der Addition und Multiplikation werden mit \(0\) bzw.\ 
        \(1\) bezeichnet.
    \end{enumerate}
\end{karte}
\begin{karte}{Nullteiler}
    Sei \(R\) ein Ring. Sind \(a,b \in R \setminus \set{0} \) 
    zwei mit \( a \cdot b = 0 \), dann heißen \(a\) und \(b\) 
    \textit{Nullteiler}. Ein Ring, der keine Nullteiler besitzt, 
    heißt Nullteilerfrei.
\end{karte}
\begin{karte}{Ringhomomorphismus}
    Seien \(R\) und \(S\) Ringe. Eine Abbildung 
    \( \abb{f}{R}{S} \) heißt \textit{Ringhomomorphismus} oder 
    \textit{Homomorphismus von Ringen}, wenn folgendes gilt:
    \begin{enumerate}
        \item \( \forall x,y \in R: f(x +_R y) = f(x) +_S f(y) \).
        \item \( \forall x,y \in R: F(x \cdot_R y) 
        = f(x) \cdot_S f(y) \).
        \item \( f(1_R) = 1_S \).
    \end{enumerate}
    Ein bijektiver Ringhomomorphismus heißt 
    \textit{Ringisomorphismus}.
\end{karte}
\begin{karte}{Definition Polynom}
    Sei \(K\) ein Körper. Ein \textit{Polynom} \(p\) über \(K\) 
    ist ein Element des\\
    \(K\)-Untervektorraums
    \[ \Abb_0(\N_0, K) := 
    \set{ \abb{f}{\N_0}{K} \;\vert \; 
    \inverse{f}(K \setminus \set{0}) \text{ endlich} } 
    \subset \Abb(\N_0, K). \]
\end{karte}
\begin{karte}{Polynom als Vektorraum}
    Sei \( X^n \) eine Abbildung, die das \(n\)-te Element auf 
    \(1\) schickt und den Rest auf \(0\).\\
    Da \( \set{X^n \;\vert \; n\in \N_0} \) eine Basis von 
    \( \Abb_0(\N_0, K) \) ist, kann man jedes 
    Polynom \(p\) eindeutig als endliche Summe schreiben 
    \[ p = a_0 X^0 + \cdots + a_d X^d \] 
    mit \textit{Koeffizienten} \( a_0 \in K \). Ist in dieser 
    Darstellung \( a_d \neq 0 \), dann bezeichnet man \(d\) als 
    \textit{Grad} von \(p\) und schreibt \( \deg(p) := d \). Man 
    setzt \( \deg(0) = -\infty \). Man schreibt 
    \[ K[X] := \Abb_0(\N_0, K). \]
\end{karte}
\begin{karte}{Definition Polynomring}
    Durch die innere Verknüpfung 
    \[ K[X] \times K[X] \rightarrow K[X], 
    (p,q) \mapsto p \cdot q, \]
    wobei \( p \cdot q \) für \( p = a_0 X^0 + a_1 X^1 
    + \cdots + a_n X^n \) und \( q = b_0 X^0 + b_1 X^1 
    + \cdots + b_m X^m \) als 
    \[ p \cdot q := c_0 X^0 + c_1 X^1 + \ldots \]
    \[ c_k = \sum_{i=0}^k a_i b_{k-i} \]
    definiert wird (beachte, dass \(c_k = 0\) für 
    \( k > n + m \)), wird \( K[X] \) zu einem kommutativen Ring. 
    Das Nullelement der Addition ist das \textit{Nullpolynom}, 
    also das Polynom, dessen Koeffizienten alle Null sind. 
    Das Einselement ist \( 1_K \cdot X^0 \).
\end{karte}
\begin{karte}{Grad bei Polynommultiplikation}
    Sei \(K\) ein Körper. 
    \begin{itemize}
        \item Für alle \( p,q \in K[X] \) gilt 
        \( \deg(p \cdot q) = \deg p + \deg q \) mit der 
        Konvention \( n + (-\infty) = -\infty \) für 
        \( n \in \N_0 \cup \set{-\infty} \).
        \item \( K[X] \) ist nullteilerfrei.
    \end{itemize}
\end{karte}
\begin{karte}{Definition Polynomfunktion}
    Sei \( p \in a_n X^n + \cdots + a_0 \in K[X] \). Die 
    durch \(p\) \textit{definierte Polynomfunktion} ist 
    \[ f_p : K \rightarrow K, x \mapsto a_n x^n + \cdots 
    + a_1 x + a_0. \]
    Man sagt, dass \( \alpha \in K \) eine \textit{Nullstelle} 
    von \(p\) ist, falls \( f_p(\alpha) = 0 \). \\
    Man schreibt auch \( p(\alpha) \) statt \(f_p(\alpha)\).\\
    Wenn \(K\) unendlich ist, dann ist 
    \( K[X] \rightarrow \Abb(K,K), \; p \mapsto f_p \) injektiv.
\end{karte}
\begin{karte}{Nullstellenlemma}
    Sei \( p \in K[X] \) und \( \alpha \) eine Nullstelle 
    von \( p \). Dann gibt es ein \( q\in K[X] \) mit 
    \( p = (X - \alpha) \cdot q \).
\end{karte}
\begin{karte}{Definition algebraische Abgeschlossenheit}
    Ein Körper \( K \), für den jedes Polynom 
    \( p \in K[X] \) mit \( \deg p \geq 1 \) mindestens eine 
    Nullstelle hat, heißt \textit{algebraisch abgeschlossen}.\\
    Jeder Körper ist ein Teilkörper eines algebraisch 
    abgeschlossenen Körpers.\\
    Die komplexen Zahlen sind algebraisch abgeschlossen.
\end{karte}
\begin{karte}{Nullstellen eines algebraisch abgeschlossenen Polynoms}
    Sei \(K\) ein algebraisch abgeschlossener Körper. Sei 
    \( p \in K[X] \). Dann existieren 
    \( \alpha_0, \alpha_1, \ldots, a_n \in K \) mit 
    \[ p = \alpha_0 (X - \alpha_1) \cdots (X - \alpha_n). \]
\end{karte}
\begin{karte}{Vielfachheit einer Nullstelle}
    Sei \( p \in K[X] \) ein Polynom mit einer Nullstelle 
    \( \alpha \). Die maximale Zahl \( n \in \N \) derart, 
    dass \( q \in K[X] \) mit 
    \[ p = {(X - \alpha)}^n \cdot q \]
    existiert, heißt \textit{Vielfachheit} der Nullstelle 
    \( \alpha \).
\end{karte}
\begin{karte}{Das charakteristische Polynom}
    Sei \( A \in M_n(K) \). Das charakteristische Polynom 
    \( \chi_A \in K[X] \) ist definiert als 
    \[ \chi_A(X) := \det(A - X \cdot I_n) \in K[X]. \]
    Die Nullstellen von \( \chi_A \) sind genau die Eigenwerte 
    von \(A\).
\end{karte}
\begin{karte}{Definition Spur}
    Sei \( A = (a_{ij}) \in M_n(K) \). Die \textit{Spur} von 
    \(A\) ist als die Summe der Diagonaleinträge von \(A\) 
    definiert: 
    \[ \Spur(A) := \sum_{i=1}^n a_{ii} \in K. \]
\end{karte}
\begin{karte}{Definition algebraische Vielfachheit}
    Sei \( \lambda \in K \) ein Eigenwert von \(A \in M_n(K)\). 
    Die \textit{algebraische Vielfachheit} des Eigenwerts 
    \( \lambda \) ist die Vielfachheit von \( \lambda \) als 
    Nullstelle von \( \chi_A \).
\end{karte}
\begin{karte}{Verhältnis geometrische und algebraische Vielfachheit}
    Sei \( \lambda \in K \) ein Eigenwert von \( A \in M_n(K) \). 
    Sei \( m_\lambda \) die algebraische Vielfachheit von 
    \( \lambda \) und sei \( d_\lambda 
    = \dim_K(\ker(A - \lambda I_n)) \) die geometrische 
    Vielfachheit von \( \lambda \). Dann gilt
    \[ d_\lambda \leq m_\lambda. \]
\end{karte}
\begin{karte}{Zusammenhang Linearfaktoren Diagonalisierbarkeit}
    Sei \( A \in M_n(K) \). Dann sind äquivalent:
    \begin{enumerate}
        \item \(A\) ist diagonalisierbar.
        \item Das charakteristische Polynom von \(A\) zerfällt 
        in Linearfaktoren und die algebraische und geometrische 
        Vielfachheit eines jeden Eigenwerts von \(A\) stimmen 
        überein.
    \end{enumerate}
    Es kann zwei Gründe geben, dass \(A\) nicht diagonalisierbar 
    ist:
    \begin{enumerate}
        \item Das charakteristische Polynom zerfällt nicht in 
        Linearfaktoren.\\
        Hier könnte man zu einem algebraisch 
        abgeschlossenen Körper übergehen (etwa von \( \R \) 
        nach \( \C \)).
        \item Das charakteristische Polynom zerfällt, aber die 
        algebraische Vielfachheit eines Eingenwerts \( \lambda \) 
        ist größer als die geometrische Vielfachheit von 
        \( \lambda \).
    \end{enumerate}
\end{karte}
\subsection*{Klassifikation}
\begin{karte}{Invarianten unter Ähnlichkeit}
    Seien \( A, B \in M_n(K) \) ähnliche Matrizen. Dann gilt: 
    \begin{itemize}
        \item Es ist \( \rk A = \rk B \).
        \item Es ist \( \spec A = \spec B \) und für alle 
        \( \lambda \in \spec A \) gilt 
        \[ \dim_K(E_\lambda(A)) = \dim_K(E_\lambda(B)). \]
        \item Es ist \( \chi_A = \chi_B \).
        \item Es ist \( \Spur A = \Spur B \).
        \item Es ist \( \det A = \det B \).
    \end{itemize}
\end{karte}
\end{document}