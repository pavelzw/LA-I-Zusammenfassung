\documentclass[main.tex]{subfiles}

\begin{document}
\section*{Klassifikation von Endomorphismen I}
\subsection*{Eigenwerte und Eigenvektoren}

\begin{karte}{Definition Eigenwert, Eigenvektor, Eigenraum\\von Endomorphismen}
    Sei \( f: V \rightarrow V \) ein Endomorphismus. Unter 
    einem \textit{Eigenvektor} von \(f\) zum \textit{Eigenwert} 
    \(\lambda \in K\) versteht man einen Vektor \(v \neq 0\) aus 
    \(V\) mit 
    \[ f(v) = \lambda v. \]
    Der \textit{Eigenraum} zu einem Eigenwert \( \lambda \) von 
    \(f\) ist definiert als 
    \[ E_\lambda(f) = \set{ v \in V \;\vert \; f(v) = \lambda v } 
    = \ker(f - \lambda \cdot \id_V). \]
\end{karte}
\begin{karte}{Definition Eigenwert, Eigenvektor, Eigenraum\\von Matrizen}
    Eigenvektoren, Eigenwerte und Eigenräume von 
    \( L(A) : K^n \rightarrow K^n, A \in M_n(K) \), bezeichnet 
    man auch als \textit{Eigenvektoren}, \textit{Eigenwerte}, 
    \textit{Eigenräume} von \(A\).
\end{karte}
\begin{karte}{Berechnung von Eigenwerten\\ohne charakteristisches Polynom}
    Sei \( f: V \rightarrow V \) ein Endomorphismus und 
    \( \lambda \in K \). Dann ist \( \lambda \) genau dann ein 
    Eigenwert von \( V \), wenn 
    \( \ker(f - \lambda \id_V) \neq 0 \). Ist 
    \( \dim_K(V) < \infty \), so ist diese Aussage äquivalent zu 
    \[ \det(f - \lambda \id_V) = 0. \]
\end{karte}
\begin{karte}{Definition Spektrum}
    Sei \(V\) ein \(K\)-Vektorraum. Sei 
    \( f: V \rightarrow V \) ein Endomorphismus. Dann 
    bezeichnen wir 
    \[ \spec_K f := \set{ \lambda \in K 
    \;\vert \; f - \lambda \id_V \text{ nicht invertierbar} } \]
    als das \textit{Spektrum} von \(f\) über \(K\). Ist 
    \( A \in M_n(K) \), so setzen wir 
    \[ \spec_K A := \spec_K(L(A)). \]
\end{karte}
\begin{karte}{Eigenvektoren von Eigenwerten}
    Sei \( \abb{f}{V}{V} \) ein Endomorphismus. Seien 
    \( \lambda_1,\ldots,\lambda_n \) verschiedene Eigenwerte 
    von \(f\). Sind \( v_1, \ldots, v_n \) Eigenvektoren von 
    \(f\) zu den Eigenwerten \( \lambda_1, \ldots, \lambda_n \), 
    so ist \( (v_1,\ldots,v_n) \) linear unabhängig.
\end{karte}
\begin{karte}{Direkte Summe der Eigenräume}
    Ist \( \abb{f}{V}{V} \) ein Endomorphismus und sind 
    \( \lambda_1,\ldots,\lambda_n \) verschiedene Eigenwerte 
    von \(f\), so ist die Summe der Untervektorräume 
    \( E_{\lambda_1}(f) + \cdots + E_{\lambda_n}(f) \) eine 
    direkte Summe.
\end{karte}
\begin{karte}{Maximalanzahl Eigenwerte}
    Ist \( \abb{f}{V}{V} \) ein Endomorphismus eines 
    \(n\)-dimensionalen Vektorraums, so besitzt \(f\) 
    höchstens \(n\) verschiedene Eigenwerte.
\end{karte}
\subsection*{Diagonalisierbarkeit}
\begin{karte}{Definition Diagonalisierbarkeit}
    Ein Endomorphismus, für den eine Basis von Eigenvektoren 
    existiert, heißt \textit{diagonalisierbar}. Eine Matrix 
    \( A \in M_n(K) \) heißt \textit{diagonalisierbar}, falls 
    \( L(A) \) diagonalisierbar ist.\\
    Ein Endomorphismus eines \(n\)-dimensionalen Vektorraums, 
    der \(n\) verschiedene Eigenwerte hat, ist diagonalisierbar. 
    Für quadratische Matrizen gilt die analoge Aussage.\\
    Eine Matrix \( A \in M_n(K) \) ist genau dann diagonalisierbar, 
    wenn sie einer Diagonalmatrix ähnlich ist.
\end{karte}
\begin{karte}{Definition Ähnlichkeit}
    Wir nennen zwei Matrizen \( A, B \in M_n(K) \) 
    \textit{ähnlich} oder \textit{zueinander konjugiert}, 
    wenn es ein \( S \in \GL_n(K) \) gibt mit 
    \[ A = \inverse{S} \cdot B \cdot S. \]
    Die durch Ähnlichkeit definierte Relation \( \sim \) 
    auf \( M_n(K) \) ist eine Äquivalenzrelation.\\
    Ähnliche Matrizen haben das gleiche Spektrum.
\end{karte}
\begin{karte}{Definition Diagonalmatrix}
    Eine Matrix \( A = (a_{ij}) \in M_n(K) \) heißt 
    \textit{Diagonalmatrix}, \\
    wenn \( a_{ij} = 0 
    \, \forall i,j \in \set{1,\ldots,n} \) mit \( i \neq j \).\\
\end{karte}
\begin{karte}{Definition geometrische Vielfachheit}
    Sei \( \lambda \in K \) ein Eigenwert eines Endomorphismus 
    \( f \). Die \textit{geometrische Vielfachheit} von 
    \( \lambda \) ist die Dimension von \( E_\lambda(f) \).\\
    Ein Endomorphismus eines \(n\)-dimensionalen Vektorraums 
    ist genau dann diagonalisierbar, wenn die Summe der 
    geometrischen Vielfachheiten aller Eigenwerte gleich 
    \(n\) ist.
\end{karte}
\end{document}