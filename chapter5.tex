\documentclass[main.tex]{subfiles}
\begin{document}

\section*{Matrizenkalkül}
\subsection*{Invertierbarkeit und Spaltenrang}

\begin{karte}{Definition Invertierbarkeit}
    Eine Matrix \(A \in M_n(K)\) ist \textit{invertierbar},
    wenn es eine Matrix \(B \in M_n(K)\) mit \(A \cdot B = I_n\) 
    und \(B \cdot A = I_n\) gibt.\\
    \(A\) ist genau dann invertierbar, wenn \(L(A): K^n \rightarrow K^n \)
    ein Isomorphismus ist.
\end{karte}
\begin{karte}{Definition Inverse einer Matrix}
    Für eine invertierbare Matrix \(A \in M_n(K)\) ist \(B \in M_n(K)\) 
    mit \\
     \(A \cdot B = B \cdot A = I_n\)
    eindeutig bestimmt, denn \(B\) entspricht unter \(L\) 
    der Umkehrabbildung von \(L(A)\). \\
    Man nennt \(A^{-1} = B\) das \textit{Inverse} von \(A\).\\
    Des Weiteren gilt: 
    \[B \cdot A = I_n \Rightarrow A \cdot B = I_n\] 
    und somit \(A^{-1} = B\).\\
    Beispiel: 
    \[ \begin{pmatrix}
        1 & 1 \\
        0 & 1
    \end{pmatrix}^{-1} = 
    \begin{pmatrix}
        1 & -1 \\
        0 & 1
    \end{pmatrix} \]
\end{karte}
\begin{karte}{Definition Spaltenrang}
    Der \textit{(Spalten-)Rang} \(\rk(A)\) einer Matrix
    \(A \in M_{m,n}(K)\) ist der Rang der linearen Abbildung 
    \(L(A): K^n \rightarrow K^m\).
    Der Spaltenrang von \(A\) ist die maximale Anzahl linear unabhängiger 
    Spaltenvektoren von \(A\).
\end{karte}
\begin{karte}{Lemma zur Dimension eines Erzeugnisses}
    Sei \(A\) eine endliche Teilmenge eines Vektorraums \(V\). \\
    Dann ist: 
    \[\dim(\langle A \rangle) = \text{max}\set{\abs B \;\vert \; 
    A \supset B \text{ linear unabhängig}}.\]
\end{karte}
\begin{karte}{Definition Zeilenrang}
    Der \textit{Zeilenrang} einer Matrix \(A \in M_{m,n}(K)\) ist 
    die maximale Anzahl linear unabhängiger Zeilenvektoren von A.
\end{karte}
\begin{karte}{Korollar zur Invertierbarkeit und Spaltenrang}
    Sei \(A \in M_n(K)\). Dann gilt:\vspace{2mm}\\
    \begin{tabular}{rl}
        & \(A\) ist invertierbar.\vspace{1mm}\\
        \( \Leftrightarrow \) & Das \(n\)-Tupel der 
        Spaltenvektoren von \(A\) ist eine Basis von \(K^n\).\vspace{1mm}\\
        \( \Leftrightarrow \) & Das \(n\)-Tupel der 
        Spaltenvektoren von \(A\) ist linear unabhängig.\vspace{1mm}\\
        \( \Leftrightarrow \) & Das \(n\)-Tupel der 
        Spaltenvektoren von \(A\) ist ein Erzeugendensystem.\vspace{1mm}\\
        \( \Leftrightarrow \) & \(\rk(A) = n\).        
    \end{tabular}
\end{karte}
\begin{karte}{Definition allgemeine lineare Gruppe}
    Die \textit{allgemeine lineare Gruppe} \( \GL_n(K) \) ist die Teilmenge 
    der invertierbaren Matrizen in \(M_n\). \\ 
    Bezüglich der Matrixmultiplikation bildet sie eine Gruppe.
\end{karte}

\subsection*{Darstellung linearer Abbildungen}  

\begin{karte}{Definition Basiswechselabbildung}
    Seien \(B = (v_1, \ldots , v_n)\) und \(C = (w_1, \ldots , w_n)\)
    Basen der \(n\)-dimensionalen \(K\)-Vektorräume \(V\) bzw. \(W\).
    Dann ist \(t_{B,C}\) der Isomorphismus, der \(v_i\) auf \(w_i\) für 
    jedes \(i \in \set{1, \ldots,n}\) abbildet: 
    \[ t_{B,C}: V \rightarrow W, v_i \mapsto w_i. \]
\end{karte}
\begin{karte}{Definition Abbildungsmatrix}
    Für eine lineare Abbildung \(f:V \rightarrow W\) zwischen den
    \(K\)-Vektorräumen \(V\) und \(W\) mit \(\dim_K(V) = n\) und 
    \(\dim_K(W) = m\) ist \(M_{B,C}(f) \in M_{m,n}(K)\) die 
    eindeutig bestimmte \textit{Matrix zu \(f\) bzgl.\ der Basen 
    B und C} von \(V\) und \(W\) mit: 
    \[ L(M_{B,C}(f)) = f_{B,C} := t_{E_m,C}^{-1} \circ f \circ t_{E_n,B} 
    \in \hom_K(K^n,K^m)\] 
\end{karte}
\begin{karte}{Satz zur Berechnung von \(M_{B,C(f)}\) nach Spaltenrang}
    Sei \(f: V \rightarrow W\) linear und seien \(B = (v_1, \ldots ,v_n)\)
    und \(C = (w_1, \ldots , w_n)\) Basen der Vektorräume \(V\) und \(W\).
    Für \(i \in \set{1,\ldots,n}\) und die eindeutig bestimmten Skalare
    \(\lambda_1,\ldots,\lambda_m\) mit
    \[f(v_i) = \lambda_1w_1 + \cdots + \lambda_m w_m\]
    gilt, dass: 
    \[{(M_{B,C}(f))}_{*i} = 
    \begin{pmatrix}
        \lambda_1 \\
        \vdots \\
        \lambda_m
    \end{pmatrix}.\]
\end{karte}
\begin{karte}{Definition Basiswechselmatrix}
    Für zwei Basen \(B\) und \(C\) eines \(n\)-dimensionalen 
    \(K\)-Vektorraums \(V\) ist die zugehörige 
    \textit{Basiswechselmatrix}:
    \[M_{B,C} := M_{B,C}(\id_V) \in \GL_n(K) \]
    Ist nun \( \abb{f}{V}{W} \) eine lineare Abbildung und 
    \(B,B'\) Basen von \(V\) und \(C,C'\) Basen von \(W\), dann gilt:
    \[ M_{B',C'}(f) = M_{C,C'} \cdot M_{B,C}(f) \cdot M_{B',B}. \]
\end{karte}
\begin{karte}{Definition transponierte Matrix}
    Ist \(A = (a_{ij}) \in M_{m,n}(K)\) dann ist die \textit{transponierte Matrix}
    \[ A^T  = (a_{ji}) \in M_{n,m}(K). \]
\end{karte}
\begin{karte}{Satz dualer Basiwechsel}
    Seien \(B\) und \(C\) Basen der endlich-dimensionalen 
    \(K\)-Vektorräume \(V\) und \(W\), und seien \(B^*\) und \(C^*\)
    die dazugehörigen dualen Basen von \(V^*\) und \(W^*\).\\
    Dann gilt für jede lineare Abbildung \( \abb{f}{V}{W} \):
    \[M_{C^*,B^*}(f^*) = {(M_{B,C}(f))}^T\]
    Der Zeilen- und Spaltenrang einer Matrix stimmen überein.
\end{karte}

\subsection*{Zeilenstufenform und LGS}
\begin{karte}{Definition Zeilenstufenform}
    Eine Matrix in Zeilenstufenform sieht wie folgt aus:
    \begin{equation*}
        A =
        \left(\mkern3mu
          \begin{array}{*{12}c}
            \multicolumn{1}{|c}{1} & (*) & (*) & (*) & (*) & (*) & (*) & \ldots & \ldots & \ldots & \ldots & (*) \\
            \cline{1-1}
            0 & \multicolumn{1}{|c}{1} & (*) & (*) & (*) & (*) & (*)  & \ldots & \ldots & \ldots & \ldots &  (*) \\
            \cline{2-3}
            0 & 0 & 0 & \multicolumn{1}{|c}{1} & (*) & (*) & (*) & \ldots & \ldots & \ldots & \ldots &  (*)\\
            \cline{4-6}
            0 & 0 & 0 & 0 & 0 & 0 & \multicolumn{1}{|c}{1} & (*) & \ldots & \ldots & \ldots &  (*) \\
            \cline{7-7}\cdashline{8-9}
            \vdots & & & & \vdots & & & & \vdots & \multicolumn{1}{:c}{} & & \vdots \\
            0 & 0 & 0 & \ldots & \ldots & \ldots & \ldots & \ldots & 0 & \multicolumn{1}{|c}{1} & \ldots & (*) \\
            \cline{10-12}
            0 & 0 & 0 & \ldots  & \ldots & \ldots & \ldots & \ldots & \ldots & \ldots & 0 &  0 \\
            \vdots & & & & \vdots & & & & \vdots & & & \vdots \\
            0 & 0 & 0 & \ldots  & \ldots & \ldots & \ldots & \ldots & \ldots & \ldots & 0 &  0
          \end{array}
         \hskip-\arraycolsep\right)
      \end{equation*}
      Die Pivotindizes \( k_1,\ldots, k_r \) sind die jeweiligen Spalten, 
      in denen in einer Zeile zum ersten Mal eine \( 1 \) vorkommt.
\end{karte}
\begin{karte}{Pivotspalten als Basis des Bilds}
    Sei \( A \in M_{m,n}(K) \) in Zeilenstufenform.
    Da die \(r\) Pivotspalten 
    \[ A_{*k_1} = A \cdot e_{k_1} = L(A)(e_{k_1}), \ldots, 
    A_{*k_r} = A \cdot e_{k_r} = L(A)(e_{k_r}) \]
    linear unabhängig sind und \( \rk A = \rk(L(A)) = r\), 
    bilden sie eine Basis des Bilds von \( L(A)\).
\end{karte}
\begin{karte}{Definition Lösungsmenge einer Matrix}
    Seien \( A \in M_{m,n}(K) \) und \( b \in K^m \). Die Menge 
    \[ V(A,b) := \set{ x \in K^n \;\vert \; A \cdot x = b } \]
    heißt die Lösungsmenge von \( A \cdot x = b \).\\
    Falls \( V(A,b) \neq \emptyset \), ist \(V\), ist 
    \( V(A,b) \) für eine \gqq{spezielle} Lösung \( w\in V(A,b) \) 
    der affine Unterraum 
    \[ V(A,b) = w + \ker(L(A)). \]
    Insbesondere ist \( V(A,0) = \ker(L(A)) \) ein Unterraum.
\end{karte}
\begin{karte}{Projektion vom Kern einer Matrix auf Nicht-Pivotvariablen}
    Sei \( A \in M_{m,n}(K) \) in Zeilenstufenform. Dann ist die 
    Projektion der Vektoren in \( V(A,0) \) auf deren Nicht-Pivotvariablen 
    \[ K^n \supset V(A,0) \rightarrow K^{n-r}, (x_i)_{i\in\set{1,\ldots,n}} 
    \mapsto (x_j)_{j\in\set{1,\ldots,n}\setminus\set{k_1,\ldots,k_r}} \]
    ein Isomorphismus.\\
    Das Urbild der kanonischen Basis von \( K^{n-r} \) unter der obigen 
    Abbildung ist somit eine Basis von 
    \[ V(A,0) = \ker(L(A)). \]
\end{karte}
\subsection*{Der Gauß-Algorithmus}
\begin{karte}{Definition Gauß-Algorithmus}
    Der \textit{Gauß-Algorithmus} überführt eine beliebige Matrix 
    \( A \in M_{m,n}(K) \) durch eine elementare Zeilenoperation 
    in Zeilenstufenform. Jede Zeilenoperation ist das Resultat der 
    Matrixmultiplikation von links mit einer \textit{Elementarmatrix} 
    aus \( \GL_m(K) \).
\end{karte}
\begin{karte}{Definition Elementarmatrix}
    Die \textit{Elementarmatrizen} \(V_{j,k}, S_{j,k}(\lambda) 
    \text{ für } \lambda \in K, \text{ und } M_j(\lambda) \text{ für }
    \lambda \in K^* \) werden wie folgt definiert:\\
    \begin{tabular}{p{1 cm}|p{3 cm}|p{3 cm}|p{1 cm}}
        Matrix \(Z\) & Lineare Abbildung\newline \(x \mapsto Z \cdot x\) & Zeilenoperation\newline \(A\leadsto Z \cdot A\) & \(\inverse{Z}\) \\ 
        \hline
        \(V_{j,k}\) & Vertauschung der Koordinaten \(j\) und \(k\) & Vertauschung der \(j\)-ten und \(k\)-ten Zeile. \newline Notation: \( (j) \leftrightarrow (k) \) & \(V_{j,k}\) \\
        \hline
        \(S_{j,k}(\lambda)\) & Addition des \(\lambda\)-fachen \newline der \(k\)-ten zur \(j\)-ten \newline Komponente, d.\ h.\ \newline Scherung in der \((j,k)\)-Koordinatenebene & Addition des \(\lambda\)-fachen\newline der \(k\)-ten Zeile zur \(j\)-ten Zeile. \newline Notation: \((j) + \lambda \cdot (k) \) & \(S_{j,k}(-\lambda)\) \\
        \hline
        \(M_j(\lambda)\) & \(j\)-te Komponente wird mit \( \lambda \in K^* \) multipliziert, d.h. Streckung der \(j\)-ten Koordinatenachse & Multiplikation der \(j\)-ten Zeile mit \(\lambda\) \newline Notation: \(\lambda \cdot (j) \) & \(M_j(\inverse{\lambda})\)
    \end{tabular}
\end{karte}
\begin{karte}{Lösungsmenge einer Matrix und elemtare Zeilenoperationen}
    Elementare Zeilenoperationen lassen die Lösungsmenge eines 
    linearen Gleichungssystems unverändert, d.\ h.\ sind 
    \( A \in M_{m,n}(K), b \in K^m \) und \( Z_1,\ldots, Z_k \in M_m(K) \)
    Elementarmatrizen und \( Z := Z_k \cdots Z_1 \), dann ist 
    \[ V(A,b) = V(Z \cdot A, Z \cdot b). \]
\end{karte}
\begin{karte}{Definition Gaußsche Normalform}
    Eine Matrix in Zeilenstufenform, in der alle Einträge der 
    Pivotspalten bis auf einen Null sind, ist in 
    \textit{Gaußscher Normalform}
\end{karte}
\begin{karte}{Inverses durch Elementarmatrizen}
    Ist \( A \in \GL_n(K) \), so gibt es Elementarmatrizen 
    \( Z_1, \ldots, Z_r \) mit \( (Z_1 \cdots Z_r)\cdot A = I_n \), 
    also \( A = \inverse{Z_1} \cdots \inverse{Z_r} \). \\
    Jede Matrix in \( \GL_n(K) \) ist ein Produkt von 
    Elementarmatrizen.
\end{karte}

\end{document}