\documentclass[main.tex]{subfiles}
\begin{document}

\section*{Matrizenkalkül}
\subsection*{Invertierbarkeit und Spaltenrang}

\begin{karte}{Definition Invertierbarkeit}
    Eine Matrix \(A \in M_n(K)\) ist \textit{invertierbar},
    wenn es eine Matrix \(B \in M_n(K)\) mit \(A \cdot B = I_n\) 
    und \(B \cdot A = I_n\) gibt.\\
    A ist genau dann invertierbar, wenn \(L(A): K^n \rightarrow K^n \)
    ein Isomorphismus ist.
\end{karte}
\begin{karte}{Definition Inverse einer Matrix}
    Für eine invertierbare Matrix \(A \in M_n(K)\) ist \(B \in M_n(K)\) 
    mit \\
     \(A \cdot B = B \cdot A = I_n\)
    eindeutig bestimmt, denn \(B\) entspricht unter \(L\) 
    der Umkehrabbildung von \(L(A)\). \\
    Man nennt \(A^{-1} = B\) das \textit{Inverse} von \(A\).\\
    Des Weiteren gilt: 
    \[B \cdot A = I_n \Rightarrow A \cdot B = I_n\] 
    und somit \(A^{-1} = B\).\\
    Beispiel: 
    \[ \begin{pmatrix}
        1 & 1 \\
        0 & 1
    \end{pmatrix}^{-1} = 
    \begin{pmatrix}
        1 & -1 \\
        0 & 1
    \end{pmatrix} \]
    %Ein Beispiel ? 
\end{karte}
\begin{karte}{Definition Spaltenrang}
    Der \textit{(Spalten-)Rang} \(\rk(A)\) einer Matrix
    \(A \in M_{m,n}(K)\) ist der Rang der linearen Abbildung 
    \(L(A): K^n \rightarrow K^m\).
    Der Spaltenrang von \(A\) ist die maximale Anzahl linear unabhängiger 
    Spaltenvektoren von \(A\).
\end{karte}
\begin{karte}{Lemma zur Dimension eines Erzeugnises}
    Sei \(A\) eine endliche Teilmenge eines Vektorraums \(V\). \\
    Dann ist: 
    \[\dim(\langle A \rangle) = \text{max}\set{\abs B \;\vert\; 
    A \supset B \text{ linear unabhängig}}.\]
\end{karte}
\begin{karte}{Definition Zeilenrang}
    Der \textit{Zeilenrang} einer Matrix \(A \in M_{m,n}(K)\) ist 
    die maximale Anzahl linear unabhängiger Zeilenvektoren von A.
\end{karte}
\begin{karte}{Korollar zur Invertierbarkeit und Spaltenrang}
    Sei \(A \in M_n(K)\). Dann gilt:
    \begin{itemize}
        \item[] \(A\) ist invertierbar.
        \item[\(\Leftrightarrow \)] Das \(n\)-Tupel der 
        Spaltenvektoren von \(A\) ist eine Basis von \(K^n\).
        \item[\(\Leftrightarrow \)] Das \(n\)-Tupel der 
        Spaltenvektoren von \(A\) ist linear unabhängig.
        \item[\(\Leftrightarrow \)] Das \(n\)-Tupel der 
        Spaltenvektoren von \(A\) ist ein Erzeugendensystem. 
        \item[\(\Leftrightarrow \)] \(\rk(A) = n\).        
    \end{itemize}
\end{karte}
\begin{karte}{Definition allgemeine lineare Gruppe}
    Die \textit{allgemeine lineare Gruppe} \(GL_n(K)\) ist die Teilmenge 
    der invertierbaren Matrizen in \(M_n\). \\ 
    Bezüglich der Matrixmultiplikation bildet sie eine Gruppe.
\end{karte}

\subsection*{Darstellung linearer Abbildungen}  

\begin{karte}{Definition Basiswechselabbildung}
    Seien \(B = (v_1, \ldots , v_n)\) und \(C = (w_1, \ldots , w_n)\)
    Basen der \(n\)-dimensionalen \(K\)-Vektorräume \(V\) bzw. \(W\).
    Dann ist \(t_{B,C}\) der Isomorphismus, der \(v_i\) auf \(w_i\) für 
    jedes \(i \in \set{1, \ldots,n}\) abbildet: 
    \[ t_{B,C}: V \rightarrow W, v_i \mapsto w_i. \]
\end{karte}
\begin{karte}{Definition Abbildungsmatrix}
    Für eine lineare Abbildung \(f:V \rightarrow W\) zwischen den
    \(K\)-Vektorräumen \(V\) und \(W\) mit \(\dim_K(V) = n\) und 
    \(\dim_K(W) = m\) ist \(M_{B,C}(f) \in M_{m,n}(K)\) die 
    eindeutig bestimmte \textit{Matrix zu \(f\) bzgl. der Basen 
    B und C} von \(V\) und \(W\) mit: 
    \[ L(M_{B,C}(f)) = f_{B,C} := t_{E_m,C}^{-1} \circ f \circ t_{E_n,B} 
    \in \hom_K(K^n,K^m)\] 
\end{karte}
\begin{karte}{Satz zur Berechnung von \(M_{B,C(f)}\) nach Spaltenrang}
    Sei \(f: V \rightarrow W\) linear und seien \(B = (v_1, \ldots ,v_n)\)
    und \(C = (w_1, \ldots , w_n)\) Basen der Vektorräume \(V\) und \(W\).
    Für \(i \in \set{1,\ldots,n}\) und die eindeutig bestimmten Skalare
    \(\lambda_1,\ldots,\lambda_m\) mit
    \[f(v_i) = \lambda_1w_1 + \cdots + \lambda_m w_m\]
    gilt, dass: 
    \[{(M_{B,C}(f))}_{*i} = 
    \begin{pmatrix}
        \lambda_1 \\
        \vdots \\
        \lambda_m
    \end{pmatrix}.\]
\end{karte}
\begin{karte}{Definition Basiswechselmatrix}
    Für zwei Basen \(B\) und \(C\) eines \(n\)-dimensionalen 
    \(K\)-Vektorraums \(V\) ist die zugehörige 
    \textit{Basiswechselmatrix}:
    \[M_{B,C} := M_{B,C}(\id_V) \in GL_n(K) \]
    Ist nun \(f: V \rightarrow W\) eine lineare Abbildung und 
    \(B,B'\) Basen von \(V\) und \(C,C'\) Basen von \(W\), dann gilt:
    \[ M_{B',C'}(f) = M_{C,C'} \cdot M_{B,C}(f) \cdot M_{B',B}. \]
\end{karte}
\begin{karte}{Definition transponierte Matrix}
    Ist \(A = (a_{ij}) \in M_{m,n}(K)\) dann ist die \textit{transponierte Matrix}
    \[ A^T  = (a_{kj}) \in M_{n,m}(K) \]
\end{karte}
\begin{karte}{Satz dualer Basiwechsel und transponierte Matrizen}
    %Besserer name ?
    Seien \(B\) und \(C\) Basen der endlich-dimensionalen 
    \(K\)-Vektorräume \(V\) und \(W\), und seien \(B^*\) und \(C^*\)
    die dazugehörigen dualen Basen von \(V^*\) und \(W^*\).\\
    Dann gilt für jede lineare Abbildung \(f: V \rightarrow W\):
    \[M_{C^*,B^*}(f^*) = {(M_{B,C}(f))}^T\]
    Der Zeilen- und Spaltenrang einer Matrix stimmen überein.
\end{karte}
\end{document}