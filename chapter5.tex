\documentclass[main.tex]{subfiles}
\begin{document}

\section*{Matrizenkalkül}
\subsection*{Invertierbarkeit und Spaltenrang}

\begin{karte}{Definition Invertierbarkeit}
    Eine Matrix \(A \in M_n(K)\) ist \textit{invertierbar},
    wenn es eine Matrix \(B \in M_n(K)\) mit \(A \cdot B = I_n\) 
    und \(B \cdot A = I_n\) gibt.\\
    A ist genau dann invertierbar, wenn \(L(A): K^n \rightarrow K^n \)
    ein Isomorphismus ist.
\end{karte}
\begin{karte}{Definition Inverse einer Matrix}
    Für eine invertierbare Matrix \(A \in M_n(K)\) ist \(B \in M_n(K)\) 
    mit \\
     \(A \cdot B = B \cdot A = I_n\)
    eindeutig bestimmt, denn B entspricht unter \(L\) 
    der Umkehrabbildung von \(L(A)\). \\
    Man nennt \(A^{-1} = B\) das \textit{Inverse} von \(A\).\\
    Des Weiteren gilt: 
    \[B \cdot A = I_n \Rightarrow A \cdot B = I_n\] 
    und somit \(A^{-1} = B\).
    %Ein Beispiel ? 
\end{karte}
\begin{karte}{Definition Spaltenrang}
    Der \textit{(Spalten-)Rang} \(\rk(A)\) einer Matrix
    \(A \in M_{m,n}(K)\) ist der Rang der linearen Abbildung 
    \(L(A): K^n \rightarrow K^m\).
    Der Spaltenrang von \(A\) ist die maximale Anzahl linear unabhängiger 
    Spaltenvektoren von \(A\).
\end{karte}
\begin{karte}{Lemma zur Dimension eines Erzeugnises}
    Sei \(A\) eine endliche Teilmenge eines Vektorraums \(V\). \\
    Dann ist: 
    \[\dim(\langle A \rangle) = \text{max}\set{\abs B \;\vert\; 
    A \supset B \text{ linear unabhängig }}\]
\end{karte}
\begin{karte}{Definition Zeilenrang}
    Der \textit{Zeilenrang} einer Matrix \(A \in M_{m,n}(K)\) ist 
    die maximale Anzahl linear unabhängiger Zeilenvektoren von A.
\end{karte}
\begin{karte}{Korollar zur Invertierbarkeit und Spaltenrang}
    Sei \(A \in M_n(K)\). Dann gilt:
    \begin{itemize}
        \item[] \(A\) ist invertierbar.
        \item[\(\Leftrightarrow\)] Das \(n\)-Tupel der 
        Spaltenvektoren von \(A\) ist eine Basis von \(K^n\).
        \item[\(\Leftrightarrow\)] Das \(n\)-Tupel der 
        Spaltenvektoren von \(A\) ist linear unabhängig.
        \item[\(\Leftrightarrow\)] Das \(n\)-Tupel der 
        Spaltenvektoren von \(A\) ist ein Erzeugendensystem. 
        \item[\(\Leftrightarrow\)] \(\rk(A) = n\).        
    \end{itemize}
\end{karte}
\begin{karte}{Definition Allgemeine lineare Gruppe}
    Die \textit{allgemeine lineare Gruppe} \(GL_n(K)\) ist die Teilmenge 
    der invertierbaren Matrizen in \(M_n\). \\ 
    Bezüglich der Matrixmultiplikation bildet sie eine Gruppe.
\end{karte}
\end{document}