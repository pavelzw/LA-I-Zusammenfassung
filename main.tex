\documentclass[a6paper,11pt,print,grid=front]{kartei}
\usepackage{style}

\begin{document}
\section*{Grundbegriffe}
\subsection*{Mengen}

\begin{karte}{Definition Menge}
    Eine Zusammenfassung gewisser unterscheidbarer Objekte, 
    der sogenannten Elemente dieser Menge.
\end{karte}
\begin{karte}{Definition Mächtigkeit}
    Die Anzahl der Elemente der Menge. Ist die Menge unendlich, 
    so ist die Mächtigkeit \( \infty \).
\end{karte}
\begin{karte}{Produkt von Mengen}
    \[ A \times B := \set{ (a,b) \;\vert \; 
    a \in A, b \in B }. \]
\end{karte}
\subsection*{Relationen}
\begin{karte}{Definition Relation}
    Eine Relation auf einer Menge \(X\)  ist eine Teilmenge 
    von \( X \times X \).
\end{karte}
\begin{karte}{Definition Äquivalenzrelation}
    Eine Äquivalenzrelation ist 
    \begin{description}
        \item[reflexiv] \( \forall \, x \in X : x \sim x \).
        \item[symmetrisch] \( x \sim y \Rightarrow y \sim x \).
        \item[transitiv] \( x \sim y \wedge y \sim z \Rightarrow 
        x \sim z \).
    \end{description}
\end{karte}
\begin{karte}{Definition Äquivalenzklasse}
    \[ [ x ] := \set{ y \in X \;\vert \; y \sim x }. \]
    Ein Element einer Äquivalenzklasse \( [x] \) heißt 
    Vertreter oder Repräsentant von \( [x] \).
\end{karte}
\begin{karte}{Definition Abbildung}
    Eine Abbildung \(f : X \rightarrow Y\) ist eine Vorschrift, 
    durch die jedem \(x \in X \) genau ein Element \(f(x) \in Y\)
    zugeordnet wird:  \\
    \[ x \mapsto f(x)\] \\
     \( f(x)\) ist das Bild von \(x\). \\
    Der Definitionsbereich von \(f\) ist \(X\). \\
    Der Zielbereich von \(f\) ist \(Y\). 
\end{karte}
\begin{karte}{Definition Bild und Urbild}
    \(f : X \rightarrow Y\) ist eine Abbildung. \\
    Ist \(A \subset X\) und \(B \subset Y\),
     dann ist das Bild von \(A\):
    \[f(A) := \set{ f(x) \;\vert \; x \in A }\]
    Auch ist \(Bild(f) := f(X)\). \\
    Das Urbild von \(B\) ist : 
    \[f^{-1}(B) := \set{x \;\vert \; f(x) \in B}\]
\end{karte}
\begin{karte}{Definition Injektivität}
Die Abbildung \(f: X \rightarrow Y\) ist injektiv, wenn \\
keine zwei Elemente von \(X\) auf dasselbe Element von \(Y\) abgebildet wird. 
\end{karte}
\begin{karte}{Definition Surjektivität}
    Die Abbildung \(f: X \rightarrow Y\) ist surjektiv, wenn \\
    \[ f(X) \; = \; Y\].
\end{karte} 
\end{document}