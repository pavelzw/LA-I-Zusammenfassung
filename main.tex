\documentclass[a6paper,11pt,print,grid=front]{kartei/kartei}
\usepackage{style}

\begin{document}
\section*{Grundbegriffe}
\subsection*{Mengen}

\begin{karte}{Definition Menge}
    Eine Zusammenfassung gewisser unterscheidbarer Objekte, 
    der sogenannten Elemente dieser Menge.
\end{karte}
\begin{karte}{Definition Mächtigkeit}
    Die Anzahl der Elemente der Menge. Ist die Menge unendlich, 
    so ist die Mächtigkeit \( \infty \).
\end{karte}
\begin{karte}{Produkt von Mengen}
    \[ A \times B := \set{ (a,b) \;\vert \; 
    a \in A, b \in B }. \]
\end{karte}
\subsection*{Relationen}
\begin{karte}{Definition Relation}
    Eine Relation auf einer Menge \(X\)  ist eine Teilmenge 
    von \( X \times X \).
\end{karte}
\begin{karte}{Definition Äquivalenzrelation}
    Eine Äquivalenzrelation ist:
    \begin{description}
        \item[reflexiv] \( \forall \, x \in X : x \sim x \).
        \item[symmetrisch] \( x \sim y \Rightarrow y \sim x \).
        \item[transitiv] \( x \sim y \wedge y \sim z \Rightarrow 
        x \sim z \).
    \end{description}
\end{karte}
\begin{karte}{Definition Äquivalenzklasse}
    \[ [ x ] := \set{ y \in X \;\vert \; y \sim x }. \]
    Ein Element einer Äquivalenzklasse \( [x] \) heißt 
    Vertreter oder Repräsentant von \( [x] \).
\end{karte}
\begin{karte}{Definition Quotientenmenge}
    Die Menge aller Äquivalenzklassen 
    \[ X / \sim := \set{ [x] \;\vert \; x \in X } \]
    ist die Quotientenmenge von \( X \) nach \( \sim \).
\end{karte}
\begin{karte}{Definition Kongruenz modulo \(n\)}
    Die Äquivalenzrelation \( \equiv_n \) auf \( \Z \) beschreibt die 
    Kongruenz modulo \(n\):
    \[ x \equiv_n y \Leftrightarrow n \;\vert \; (x-y) 
    \ (\text{\glqq{}} n \text{ teilt } x-y \text{\grqq{}}). \] 
\end{karte}
\begin{karte}{Quotientenmenge Kongruenz modulo \(n\)}
    Die Quotientenmenge wird mit \( \Z / n\Z \) bezeichnet. \\
    Sie enthält genau \(n\) Äquivalenzklassen, nämlich 
    \( [0], \ldots, [n-1] \).
\end{karte}
\begin{karte}{Definition teilweise Ordnung}
    Eine teilweise Ordnung \( \prec \) ist:
    \begin{description}
        \item[reflexiv] \( \forall \, x \in X : x \prec x \).
        \item[antisymmetrisch] \( x \prec y \wedge y \prec x 
        \Rightarrow y = x \).
        \item[transitiv] \( x \prec y \wedge y \prec z \Rightarrow 
        x \prec z \).
    \end{description}
    Eine Menge \(X\) mit der teilweisen Ordnung \( \prec \) nennt 
    man auch teilweise geordnete Menge.\\
    Gilt für je zwei Elemente \( x,y \) der Menge \( x \prec y \) 
    oder \( y \prec x \), so ist \(X\) streng geordnet.
\end{karte}
\begin{karte}{Definition obere Schranke}
    Ein Element \( s \in X \) ist eine obere Schranke für 
    \( T \subset X \), wenn \( z \prec s \,\forall \, z \in T \) 
    gilt.\\
    Ein Element \( m \in X \) ist maximal, wenn aus 
    \( m \prec z \) stets \( m = z \) folgt.
\end{karte}

\end{document}