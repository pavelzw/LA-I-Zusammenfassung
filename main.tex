\documentclass[a6paper,11pt,print,grid=front]{kartei}
\usepackage{style}

\begin{document}
\section*{Grundbegriffe}
\subsection*{Mengen}
\begin{karte}{Definition Menge}
    Eine Zusammenfassung gewisser unterscheidbarer Objekte, 
    der sogenannten Elemente dieser Menge.
\end{karte}
\begin{karte}{Definition Mächtigkeit}
    Die Anzahl der Elemente der Menge. Ist die Menge unendlich, 
    so ist die Mächtigkeit \( \infty \).
\end{karte}
\begin{karte}{Produkt von Mengen}
    \[ A \times B := \set{ (a,b) \;\vert \; 
    a \in A, b \in B }. \]
\end{karte}
\subsection*{Relationen}
\begin{karte}{Definition Relation}
    Eine Relation auf einer Menge \(X\)  ist eine Teilmenge 
    von \( X \times X \).
\end{karte}
\begin{karte}{Definition Äquivalenzrelation}
    Eine Äquivalenzrelation ist 
    \begin{description}
        \item[reflexiv] \( \forall \, x \in X : x \sim x \).
        \item[symmetrisch] \( x \sim y \Rightarrow y \sim x \).
        \item[transitiv] \( x \sim y \wedge y \sim z \Rightarrow 
        x \sim z \).
    \end{description}
\end{karte}
\begin{karte}{Definition Äquivalenzklasse}
    \[ [ x ] := \set{ y \in X \;\vert \; y \sim x }. \]
    Ein Element einer Äquivalenzklasse \( [x] \) heißt 
    Vertreter oder Repräsentant von \( [x] \).
\end{karte}
\subsection*{Abbildungen}
\begin{karte}{Definition Abbildung}
    Eine Abbildung \(f : X \rightarrow Y\) ist eine Vorschrift, 
    durch die jedem \(x \in X \) genau ein Element \(f(x) \in Y\)
    zugeordnet wird:  \\
    \[ x \mapsto f(x)\] \\
     \( f(x)\) ist das Bild von \(x\). \\
    Der Definitionsbereich von \(f\) ist \(X\). \\
    Der Zielbereich von \(f\) ist \(Y\). 
\end{karte}
\begin{karte}{Definition Bild und Urbild}
    \(f : X \rightarrow Y\) ist eine Abbildung. \\
    Ist \(A \subset X\) und \(B \subset Y\),
     dann ist das Bild von \(A\):
    \[f(A) := \set{ f(x) \;\vert \; x \in A }\]
    Auch ist \(Bild(f) := f(X)\). \\
    Das Urbild von \(B\) ist : 
    \[f^{-1}(B) := \set{x \;\vert \; f(x) \in B}\]
\end{karte}
\begin{karte}{Definition Injektivität}
Die Abbildung \(f: X \rightarrow Y\) ist injektiv, wenn \\
keine zwei Elemente von \(X\) auf dasselbe Element von \(Y\) abgebildet wird. 
\end{karte}
\begin{karte}{Definition Surjektivität}
    Die Abbildung \(f: X \rightarrow Y\) ist surjektiv, wenn \\
    \[ f(X) \; = \; Y\].
\end{karte} 
\begin{karte}{Definition Bijektivität}
    Ist eine Abbildung injektiv und surjektiv, dann ist sie bijektiv. 
\end{karte}
\begin{karte}{Definition Identität}
    Die Abbildung \(id_A : A \rightarrow A, id_A(a) = a\) ist die Identität.
\end{karte}
\begin{karte}{Defintion kanonische Abbildung}
    Ist \(\sim\) eine Äquivalenzrelation auf \(X\). \\
    Die surjektive Abbildung \(X \rightarrow X / \sim, \; x \mapsto [x]\), 
    ist die kanonische Projektion oder Quotientenabbildung. 
\end{karte}
\begin{karte}{Definition Wohldefiniertheit}
    Eine Abbildung \(f: X \rightarrow Y\) ist wohldefiniert, \\
    wenn für eine Äquivalenzrelation 
    \[\sim : X / \sim Y, [x] \mapsto f(x)\] 
    gilt, dass aus \(x \sim x\prime \) stets \(f(x) = f(x \prime)\) folgt. 
\end{karte}
\begin{karte}{Definition Komposition}
    Für \(f: X \rightarrow Y\) und \(g: Y \rightarrow Z\) ist 
    die Komposition :
    \[g \circ f : X \rightarrow Z,\; x \mapsto g(f(x))\]
\end{karte}
\begin{karte}{Definition Kommutatives Diagramm}
    Wenn in einem Diagramm zu je zwei Mengen alle Abbildungen 
    (auch Kompositionen), die die eine Menge in die andere abbilden,
    übereinstimmen, dann nennt man das Diagramm kommutativ. 
\end{karte}
\begin{karte}{Definition Inverses}
    Für eine bijektive Abbildung \(f: X \rightarrow Y\) ist
    \[f^{-1}: Y \rightarrow X, f(x) \mapsto x\] 
    die Umkehrabbildung oder Inverses.
\end{karte}
\begin{karte}{Definition Einschränkung}
    Eine Einschränkung von \(f: X \rightarrow Y\) 
    auf \(A \subset X\) ist : \\
    \[f \vert_A : A \rightarrow Y, x \mapsto f(x)\]
\end{karte}
\subsection*{Körper}
\begin{karte}{Definition Innere Verknüpfung}
    Eine Abbildung \(f: M x M \rightarrow M\) 
    ist eine innere Verknüpfung auf der Menge M.
\end{karte}
\begin{karte}{Defintion Gruppe}
    Eine Gruppe \( (G,*) \) besteht aus einer Menge 
    \(G\) und einer inneren Verknüpfung \(*\) auf \(G\).\\
    Eine Gruppe hat folgende Eigenschaften: 
    \begin{description}
        \item[Sie ist assoziativ :] \(\forall \; a,b,c \in G : (a * b) * c = a * (b * c)\).
        \item[Es gibt ein neutrales Element:] \(\forall a \in G : \exists e \in G : e*a = a*e = a\).
        \item[Jedes Element hat ein Inverses:] Zu jedem \(a \in G : \exists b \in G : a * b = b * a = e\). 
    \end{description}
    Gilt insbesondere \(\forall a,b \in G : a * b = b * a \), so ist 
    \((G,*)\) abelsch oder kommutativ. 
\end{karte}
\begin{karte}{Definition Körper}
    Ein Körper \((K,+,\cdot)\) besteht aus einer Menge \(K\) und zwei 
    inneren Verknüpfungen. \\
    Ein Körper hat folgende Eigenschaften: 
    \begin{description}
        \item[Bzgl. Addition abelsch]\((K,+)\) ist abelsch.
        \item[Bzgl. Multiplikation abelsch]\((K*,\cdot)\) ist abelsch mit \(K* = K \ \set{0}\).
        \item[Distributiv]\(\forall a,b,c \in K : a \cdot (b + c) + a \cdot b + a \cdot c\).   
    \end{description}
\end{karte}
\begin{karte}{Definition Teilkörper}
    Für \(L \subset K\) und \(K,+,\cdot\) ist L ein Teilkörper, 
    wenn sich die Addition \(+\) und die Multiplikation \(\cdot\) 
    zu inneren Verknüpfungen auf \(L\) einschränken, sodass \(L\)
    selbst zu einem Körper wird.
\end{karte}
\end{document}