\documentclass[main.tex]{subfiles}
\begin{document}

\section*{Lineare Abbildungen}
\subsection*{Der Begriff der linearen Abbildung}

\begin{karte}{Definition Homomorphismus}
    Eine Abbildung \( f: V \rightarrow W \) zwischen 
    \(K\)-Vektorräumen heißt \textit{\(K\)-linear} oder 
    \textit{Homomorphismus}, wenn 
    \[ f(x+y) = f(x) + f(y) \text{ und } f(\lambda x) = \lambda f(x) \]
    für alle \( x,y\in V \) und \( \lambda \in K \) gilt.\\
    Die Menge aller Homomorphismen von \(V\) nach \(W\) wird mit 
    \( \hom_K(V,W) \) bezeichnet.\\
    Die Teilmenge \( \hom_K(V,W) \subset \mathrm{Abb}(V,W) \) ist ein 
    \(K\)-Untervektorraum.
\end{karte}
\begin{karte}{Definition Biliniarität}
    Es seien \( U,V,W \) \(K\)-Vektorräume. Eine Abbildung 
    \( f: U \times V \rightarrow W \) ist \textit{\(K\)-bilinear}, 
    wenn für jedes \( u_0 \in U \) und jedes \( v_0 \in V \) die 
    folgenden Abbildungen linear sind:
    \begin{align*}
        V \rightarrow W, \,&v\mapsto f(u_0, v)\\
        U \rightarrow W, \,&u\mapsto f(u,v_0).
    \end{align*}
\end{karte}
\begin{karte}{Kompositionssatz}
    Die Komposition zweier linearer Abbildungen ist linear.\\
    Die Komposition ist als Abbildung 
    \[ \hom_K(V,W) \times \hom_K(U,V) \rightarrow 
    \hom_K(U,W),\; (f,g) \mapsto f\circ g \]
    bilinear.
\end{karte}
\begin{karte}{Satz der linearen Umkehrabbildung}
    Ist \( f: V \rightarrow W \) linear und bijektiv, so ist 
    auch \( \inverse{f} \) linear.
\end{karte}
\begin{karte}{Definition Endomorphismus, Isomorphismus, Automorphismus}
    \begin{itemize}
        \item Ein \textit{Endomorphismus} ist eine lineare Abbildung von \(V\) 
        nach \(V\).
        \item Ein \textit{Isomorphismus} ist eine bijektive lineare Abbildung.
        \item Ein \textit{Automorphismus} ist ein Isomorphismus und ein Endomorphismus.
    \end{itemize}
\end{karte}
\begin{karte}{Definition Bild}
    Sei \( f: V \rightarrow W \) eine lineare Abbildung.\\
    Für das Bild von \(f\) gilt:
    \[ \Bild(f) := f(V). \]
    Das Bild von \(f\) ist ein Untervektorraum.
\end{karte}
\begin{karte}{Definition Kern}
    Der \textit{Kern} ist wie folgt definiert:
    \[ \ker(f) := \inverse{f}(\set{0}). \]
    Der Kern von \(f\) ist ein Untervektorraum.
\end{karte}
\begin{karte}{Geordnete Basis}
    Eine \textit{geordnete Basis} von \(V\) ist ein Tupel \( (v_1,\ldots,v_n) \) 
    an linear unabhängigen Vektoren, die \(V\) erzeugen.
\end{karte}

\end{document}