\documentclass[main.tex]{subfiles}
\begin{document}

\section*{Lineare Abbildungen}
\subsection*{Der Begriff der linearen Abbildung}

\begin{karte}{Definition Homomorphismus}
    Eine Abbildung \( f: V \rightarrow W \) zwischen 
    \(K\)-Vektorräumen heißt \textit{\(K\)-linear} oder 
    \textit{Homomorphismus}, wenn 
    \[ f(x+y) = f(x) + f(y) \text{ und } f(\lambda x) = \lambda f(x) \]
    für alle \( x,y\in V \) und \( \lambda \in K \) gilt.\\
    Die Menge aller Homomorphismen von \(V\) nach \(W\) wird mit 
    \( \hom_K(V,W) \) bezeichnet.\\
    Die Teilmenge \( \hom_K(V,W) \subset \mathrm{Abb}(V,W) \) ist ein 
    \(K\)-Untervektorraum.
\end{karte}
\begin{karte}{Definition Biliniarität}
    Es seien \( U,V,W \) \(K\)-Vektorräume. Eine Abbildung 
    \( f: U \times V \rightarrow W \) ist \textit{\(K\)-bilinear}, 
    wenn für jedes \( u_0 \in U \) und jedes \( v_0 \in V \) die 
    folgenden Abbildungen linear sind:
    \begin{align*}
        V \rightarrow W, \,&v\mapsto f(u_0, v)\\
        U \rightarrow W, \,&u\mapsto f(u,v_0).
    \end{align*}
\end{karte}
\begin{karte}{Kompositionssatz}
    Die Komposition zweier linearer Abbildungen ist linear.\\
    Die Komposition ist als Abbildung 
    \[ \hom_K(V,W) \times \hom_K(U,V) \rightarrow 
    \hom_K(U,W),\; (f,g) \mapsto f\circ g \]
    bilinear.
\end{karte}
\begin{karte}{Satz der linearen Umkehrabbildung}
    Ist \( f: V \rightarrow W \) linear und bijektiv, so ist 
    auch \( \inverse{f} \) linear.
\end{karte}
\begin{karte}{Definition Endomorphismus, Isomorphismus, Automorphismus}
    \begin{itemize}
        \item Ein \textit{Endomorphismus} ist eine lineare Abbildung von \(V\) 
        nach \(V\).
        \item Ein \textit{Isomorphismus} ist eine bijektive lineare Abbildung.
        \item Ein \textit{Automorphismus} ist ein Isomorphismus und ein Endomorphismus.
    \end{itemize}
\end{karte}
\begin{karte}{Definition Bild}
    Sei \( f: V \rightarrow W \) eine lineare Abbildung.\\
    Für das Bild von \(f\) gilt:
    \[ \Bild(f) := f(V). \]
    Das Bild von \(f\) ist ein Untervektorraum.
\end{karte}
\begin{karte}{Definition Kern}
    Der \textit{Kern} ist wie folgt definiert:
    \[ \ker(f) := \inverse{f}(\set{0}). \]
    Der Kern von \(f\) ist ein Untervektorraum.
\end{karte}
\begin{karte}{Definition Projektion}
    Eine \textit{Projektion} ist eine lineare Abbildung \(p: V \rightarrow V\)
    mit \(p \circ p = p\). Weiter gilt: 
    \begin{itemize}
        \item \(\id_V - p\) ist ebenfalls eine Projektion.
        \item \(V = \Bild p \oplus \Bild (\id_V - p) \).
    \end{itemize}     
\end{karte}
\begin{karte}{Definition Geordnete Basis}
    Eine \textit{geordnete Basis} von \(V\) ist ein Tupel \( (v_1,\ldots,v_n) \) 
    an linear unabhängigen Vektoren, die \(V\) erzeugen.
\end{karte}
\subsection*{Basen und Dimensionen}
\begin{karte}{Lineare Fortsetzung}
    Seien \(V\) und \(W\) Vektorräume und \( (v_1,\ldots,v_n) \) 
    eine geordnete Basis von \(V\). Dann gibt es zu jedem 
    \(n\)-Tupel \((w_1,\ldots,w_n)\) von Vektoren in \(W\) 
    genau eine lineare Abbildung \( \abb{f}{V}{W} \) mit 
    \( f(v_i) = w_i \) für alle \( i \in \set{1,\ldots,n} \).
\end{karte}
\begin{karte}{Eigenschaften des Bilds der linearen Abbildung}
    Seien \(V\) und \(W\) \(K\)-Vektorräume und 
    \((v_1,\ldots,v_n)\) eine Basis von \(V\). Sei 
    \( \abb{f}{V}{W} \) eine lineare Abbildung. Dann gilt 
    \begin{itemize}
        \item \( f \surjektiv \Leftrightarrow 
        (f(v_1),\ldots,f(v_n)) \) Erzeugendensystem.
        \item \( f \injektiv \Leftrightarrow 
        (f(v_1),\ldots,f(v_n)) \) linear unabhängig.
        \item \( f \bijektiv \Leftrightarrow
        (f(v_1),\ldots,f(v_n)) \) Basis.
    \end{itemize}
\end{karte}
\begin{karte}{Isomorphiesatz}
    Je zwei \(K\)-Vektorräume der gleichen Dimension 
    \( n \in \N_0 \) sind isomorph.\\
    Isomorphe Vektorräume haben die gleiche Dimension.
\end{karte}
\begin{karte}{Universelle Eigenschaft des Quotienten}
    Sei \( U \subset V \) ein Untervektorraum und 
    \( \abb{p}{V}{V/U} \) die kanonische Projektion. 
    Sei \( \abb{f}{V}{W} \) eine lineare Abbildung 
    mit \( U \subset \ker f \). Dann gibt es genau eine 
    lineare Abbildung \( \abb{\bar{f}}{V/U}{W} \), 
    für die gilt:
    \begin{center}
    \begin{tikzcd}
        V \arrow{r}{f} \arrow{d}{p} & W \\
        V/U \arrow[dashrightarrow]{ur}{\bar{f}}
    \end{tikzcd}
    \end{center}
    \[ f = \bar{f} \circ p. \]
    Wir haben zueinander inverse Isomorphismen:
    \begin{center}
    \begin{tikzcd}
        \hom_K(V/U, W) \arrow[bend left]{r}{g \mapsto g \circ p} & \set{ f \in \hom_K(V,W) \;\vert \; U \subset \ker f } \arrow[bend left]{l}{\bar{f} \mapsfrom f }
    \end{tikzcd}
    \end{center}
\end{karte}
\begin{karte}{Homomorphiesatz}
    Eine lineare Abbildung \(f: V \rightarrow W\) 
    induziert einen Isomorphismus
    \[ V/\ker(f) \xrightarrow{\overset{\sim}{=}} f(V). \]
\end{karte}
\begin{karte}{Definition Rang einer linearen Abbildung}
    Sei \(f : V \Rightarrow W\) eine lineare Abbildung. 
    \(\rk(f) := \dim \Bild(f) \) ist der Rang von \(f\).
\end{karte}
\begin{karte}{Dimensionsformel einer linearen Abbildung}
    Für eine lineare Abbildung \( f: V \rightarrow W \) 
    gilt 
    \[ \dim V = \rk f + \dim(\ker f). \]
    Eine lineare Abbildung zwischen Vektorräumen 
    der gleichen Dimension ist genau dann surjektiv, wenn sie 
    injektiv ist.    
\end{karte}
\subsection*{Matrizen}
\begin{karte}{Definition Matrix}
    Eine \textit{\(m \times n\)-Matrix} über \(K\) ist 
    eine über \( \set{1,\ldots,m} \times \set{1,\ldots,n} \) 
    indizierte Familie in \(K\), die wir nach folgendem Schema 
    anordnen:
    \[ \begin{pmatrix}
        a_{11} & \cdots & a_{1n}\\
        \vdots & \ddots & \vdots \\
        a_{m1} & \cdots & a_{mn}
    \end{pmatrix}. \]
    Die \( a_{ij} \in K \) nennt man auch 
    \textit{Koeffizienten} der Matrix. Man notiert die obige 
    Matrix auch als \( (a_{ij}) \). \\
    Die Menge aller \( m \times n \)-Matrizen über \(K\) wird 
    mit \( M_{m,n}(K) \) (oder mit \(M_n(K)\) für \(n=m\)) 
    bezeichnet. Wir identifizieren \( M_{m,1} \) mit \(K^m\) 
    und \( M_{1,1} \) mit \(K\).
\end{karte}
\begin{karte}{Definition Zeilen und Spalten einer Matrix}
    Sei \(k \in \set{1, \ldots, m}\) und \(l \in \set{1, \ldots , n}\).
    Weiter sei \(A = (a_{ij}) \in M_{m,n}(K)\). 
    Die \(k\)-te Zeile von \(A\) ist: 
    \[A_{k*} := (a_{k1},  \ldots, a_{kn}) \in M_{1,n}(K)\]
    Die \(l\)-te Spalte von \(A\) ist:
    \[A_{*l} := (a_{1l}, \ldots , a_{ml}) \in M_{m,1}(K)\]                        
\end{karte}
\begin{karte}{Definition Matrixmultiplikation}
    \begin{itemize}
        \item Für \(A = (a_{1i}) \in M_{1,n}(K)\) und 
        \(B = (b_{i1}) \in M_{n,1}\) definiere: 
        \[A \cdot B := \sum_{i=1}^n a_{1i}b_{i1} \in K\]
        \item Für \(A \in M_{n,p}(K)\) und \(B \in M_{p,q}(K)\) definiere
        \[A \cdot B = {(A_{i*} \cdot B_{*k})}_{\substack{i=1,\ldots,n \\ k=1,\ldots,q}} 
        \in M_{n,q}(K) \]
    \end{itemize}
\end{karte}

\begin{karte}{Einheitsmatrix}
    Sei \( I_n \in M_n(K) \) die Matrix, deren \(i\)-te Spalte 
    der \(i\)-te Vektor der kanonischen Basis ist, d.\ h.\ 
    \[ I_n = 
    \begin{pmatrix}
        1 & 0 & 0 & \cdots & 0 \\
        0 & 1 & 0 & \cdots & 0 \\
        0 & 0 & 1 & \cdots & 0 \\
        \vdots & \vdots & \vdots & \ddots & \vdots \\
        0 & 0 & 0 & \cdots & 1 \\
    \end{pmatrix} \]
    \(I_n\) heißt \textit{\( n \times n \)-Einheitsmatrix}. 
    Mithilfe des Kronecker-Deltas 
    \[ \delta_{j,k} = \begin{cases}
        1 & \text{falls } j = k\\
        0 & \text{sonst}
    \end{cases} \]
    können wir \(I_n\) auch als \( (\delta_{j,k}) \) schreiben.
\end{karte}
\begin{karte}{Definition Punktweise Addition und skalare Multiplikation von Matrizen}
    Sind \(A = (a_{ij})\),\(B = (b_{ij}) \in M_{m,n}(K) \) 
    und \(\lambda \in K\), so definiert man: 
    \[A + B := (a_{ij} + b_{ij}) \in M_{m,n}(K) \]
    \[ \lambda \cdot A := (\lambda a_{ij}) \in M_{m,n}(K) \]
    Insbesondere \({(A + B)}_{*I} = A_{*i} + B_{*i}\) als Vektoren in 
    \(K^m = M_{m,1}\).
    Dann ist \(M_{m,n}(K)\) ein \(K\)-Vektorraum der Dimension \(mn\) 
    mit den oberen inneren verknüpfungen.    
\end{karte}
\begin{karte}{Isomorphismus Matrix und lineare Abbildung}
    Die Abbildung 
    \[ \abb{L}{M_{m,n}(K)}{\hom_K(K^n,K^m)},\; 
    L(A)(x) := A x \]
    ist ein Isomorphismus von \(K\)-Vektorräumen.\\
    Insbesondere gilt \( \dim(\hom_K(K^n,K^m)) = mn \).
\end{karte}
\begin{karte}{Satz zum Verhältnis Komposition und Matrixmultiplikation}
    Für \(m,n,p \in \N \) kommutiert das folgende Diagramm, d.h. 
    Matrixmultiplikation und Komposition linearer Abbildungen 
    entsprechen sich: 
    \begin{center}
    \begin{tikzcd}
        \hom_K(K^p, K^m) \times \hom_K(K^n, K^p) \arrow{rr}{(f,g) \mapsto f \circ g} && \hom_K(K^n, K^m) \\
        \\
        M_{m,p}(K) \times M_{p,n}(K) \ar{uu}{\substack{L \times L\\\overset{\sim}{=}}} \ar{rr}{(A,B) \mapsto A \cdot B} && M_{m,n}(K) \ar{uu}{L \; \overset{\sim}{=}}
    \end{tikzcd}
    \end{center}
\end{karte}
\begin{karte}{Rechenregeln Matrixmultiplikation}
    Seien \( m,n,p,q \in \N \).
    \begin{itemize}
        \item \( A \cdot I_n = A \) und \( I_m \cdot A = A \) 
        für jedes \( A \in M_{m,n}(K) \).
        \item Für alle \( A, A' \in M_{m,n}(K) \) und alle 
        \( B,B' \in M_{n,p}(K) \) gelten 
        \[ (A+A')\cdot B = A\cdot B + A' \cdot B \text{ und } 
        A \cdot (B + B') = A \cdot B + A \cdot B'. \]
        \item Für alle \( A \in M_{m,n}(K), B \in M_{n,p}(K), 
        C \in M_{p,q}(K) \) gilt 
        \[ A \cdot (B \cdot C) = (A \cdot B) \cdot C. \]
        \item Für alle \( \lambda \in K, A \in M_{m,n}(K) \) 
        und \( B \in M_{n,p}(K) \) gilt 
        \[ \lambda \cdot (A \cdot B) 
        = (\lambda \cdot A) \cdot B
        = A \cdot (\lambda \cdot B). \]
    \end{itemize}
\end{karte}
\subsection*{Dualräume}
\begin{karte}{Definition Dualraum}
    Der \textit{Dualraum} eines \(K\)-Vektorraums \(V\) ist der 
    \(K\)-Vektorraum 
    \[ V^* := \hom_K(V,K). \]
    Die Elemente von \(V^*\) heißen auch \textit{\(K\)-Linearformen}.
\end{karte}
\begin{karte}{Definition duale Abbildung}
    Sei \( \abb{f}{V}{W} \) eine lineare Abbildung.\\
    Dann ist die zu \(f\) \textit{duale Abbildung}
    \[ \abb{f^*}{W^*}{V^*}, \phi \mapsto \phi \circ f \]
    ebenfalls linear.
    Ist \(g: W \rightarrow U\) eine weitere lineare Abbildung,
    so ist \({(g \circ f)}^* = f^* \circ g^* \).    
\end{karte}
\begin{karte}{Definition duale Basis}
    Sei \( (v_1, \ldots, v_n) \) eine Basis eines 
    \(n\)-dimensionalen Vektorraums \(V\). Dann ist 
    die lineare Abbildung 
    \[ \abb{f}{V^*}{K^n}, 
    \phi \mapsto (\phi(v_1), \ldots, \phi(v_n)) \]
    ein Isomorphismus. Sei \( v_i^* := \inverse{f}(e_i) \). 
    Man nennt \( (v_1^*, \ldots, v_n^*) \) die zu 
    \( (v_1, \ldots, v_n) \) \textit{duale Basis}.\\
    Insbesondere gilt \( \dim V^* = \dim V \).\\
    Explizit haben wir 
    \[ v_i^*(v_j) = \begin{cases}
        1, & i=j \\
        0, & i \neq j
    \end{cases}. \]
\end{karte}
\begin{karte}{Definition kanonische duale Abbildung}
    Die \textit{kanonische Abbildung} \( V \rightarrow V^{**} \),
    die einen Vektor \(v\) auf die Auswertung bei \(v\), also auf die 
    Linearform \( V^* \rightarrow K,\, \phi \mapsto \phi(v)\), abbildet,
    ist injektiv und, falls \(\dim(V) < \infty \), ein Isomorphismus.
\end{karte}
\begin{karte}{Der kanonische Isomorphismus im Dualraum}
    Sei \( \abb{f}{V}{W} \) linear. Es gibt einen kanonischen 
    Isomorphismus 
    \[ \ker(f^*) \overset{\sim}{=} {(W/\Bild(f))}^*. \]
\end{karte}
\begin{karte}{Äquivalenzsatz der dualen Abbildung}
    Für eine lineare Abbildung \( \abb{f}{V}{W} \) 
    zwischen endlich-dimensionalen Vektorräumen gelten:
    \begin{itemize}
        \item \( f \injektiv \Leftrightarrow f^* \surjektiv \).
        \item \( f \surjektiv \Leftrightarrow f^* \injektiv \).
        \item \( \rk f = \rk f^* \).
    \end{itemize}
\end{karte}
\end{document}