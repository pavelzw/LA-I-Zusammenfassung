\documentclass[main.tex]{subfiles}

\begin{document}

\section*{Vektorräume}
\subsection*{Der Begriff des Vektorraums}
\begin{karte}{Definition Vektorraum}
    Ein \( K \)-Vektorraum ist eine Menge \(V\) mit der inneren 
    Verknüpfung Vektoraddition 
    \[ V \times V \rightarrow V, (v,w) \mapsto v + w \]
    und der skalaren Multiplikation 
    \[ K \times V \rightarrow V, (\alpha, v) \mapsto 
    \alpha \cdot v. \]
    so dass gilt: 
    \begin{description}
        \item[Abelsche Gruppe der Addition]
        \( (V,+) \) ist eine abelsche Gruppe.  
        \item[Distributivität von Addition und skalarer Multiplikation] \hfill \\ 
        \( \forall \alpha, \beta \in K\) und \( v, w \in V\) gelten: \\
        \( (\alpha + \beta) \cdot v =  \alpha \cdot v + \beta \cdot v\)
        und 
        \( \alpha \cdot ( v + w) = \alpha \cdot v + \alpha \cdot w\).
        \item[Assoziativität der skalaren Multiplikation] \hfill \\
        \( \forall \alpha, \beta \in K\) und \( v \in V\) gelten:
        \((\alpha \beta) \cdot v = \alpha \cdot (\beta \cdot v)\).
        \item[Existenz eines neutralen Elements]
        \( \forall v \in V : \exists 1_K \cdot v = v\).
    \end{description}
\end{karte}
\begin{karte}{Definition Untervektorraum}
    Ein Untervektorraum \(U\) vom \(K\)-Vektorraum \(V\) ist eine nicht-leere
    Teilmenge \(U \subset V\), für die gilt: 
    \[\alpha \in K \text {und } u, v \in U \Rightarrow \alpha \cdot v \in U
    \text{ und } u + v \in U.\]
    Ein \(K\)-Untervektorraum ist ein \(K\)-Vektorraum 
    (bez.\ der eingeschränkten Addition und skalaren Multiplikation).
\end{karte}
\begin{karte}{Definition affiner Unterraum}
    Ist \(A\) ein affiner Unterraum \(A \subset V\) eines Vektorraums \(V\),
    so existieren ein Untervektorraum \(U \in V\) und ein Vektor \(v_0 \in V\) 
    und es gilt: 
    \[A = v_0 + U := \set{v_0 + v \;\vert \; v \in U}\] 
    Ist \(U = \set{\lambda \cdot w \;\vert \; \lambda \in K}\) 
    mit \(w \in V \setminus \set{0}\) heißt \(A\) affine Gerade. \\
    Ein affiner Untervektorraum \(A \subset V\) ist genau 
    dann ein Untervektorraum, wenn \(0 \in A\).
\end{karte}
\subsection*{Lineare Unabhängigkeit und Basen}
\begin{karte}{Definition Linearkombination}
    Für eine endliche Indexmenge \(J\), eine Familie 
    \( {(\lambda_j)}_{j \in J} \) von Skalaren in 
    \(K\) und eine Familie von Vektoren 
    \( {(v_j)}_{j \in J} \) in \(V\), ist \(v\) eine 
    Linearkombination der Vektoren \(v_j, j \in J\), falls: 
    \[v = \sum_{j \in J} \lambda_j v_j. \]
\end{karte}
\begin{karte}{Definition Erzeugnis}
    Das Erzeugnis von \(\langle S \rangle \) ist die Menge 
    aller Linearkombinationen von Vektoren aus \(S \subset V\). \\
    Man definiert \(\langle \emptyset \rangle : = \set{0}\). \\
    Das Erzeugnis von \(S\) ist ein Untervektorraum von \(V\), 
    der \(S\) enthält.
\end{karte}
\begin{karte}{Definition Erzeugendensystem}
    \(S\) ist ein Erzeugendensystem von \(V\), wenn \(\langle S \rangle = V\) \\
    Man sagt, dass \(S\) den Vektorraum \(V\) erzeugt.
\end{karte}
\begin{karte}{Definition Lineare Unabhängigkeit}
    Eine Teilmenge \(S \subset V\) ist genau dann linear unabhängig,
    wenn für jede endliche Teilmenge \(E \subset S\) 
    und jede Familie von Skalaren \( {(\lambda_v)}_{v \in E}\) gilt: 
    \[ 0 = \sum_{v \in E} \lambda_v v \Rightarrow \lambda_v = 0 
    \text{ für jedes } v \in E. \]
    Gilt dies nicht, dann ist \(S\) linear abhängig.
\end{karte}
\begin{karte}{Definition Basis}
    Eine Basis ist eine linear unabhängige Teilmenge von \(V\), die
    \(V\) erzeugt. \\
    Eine Menge \(\set{v_1, \ldots ,v_n}\) von \(n\) paarweise verschiedenen 
    Vektoren in \(V\) ist genau dann eine Basis, wenn es für jeden \(v \in V\)
    eindeutig bestimmten Skalaren \(\lambda_1, \ldots , \lambda_n \in K\)
    so, dass: 
    \[v = \lambda_1 v_1 + \cdots + \lambda_n v_n.\]
    Jeder Vektorraum besitzt eine Basis. 
\end{karte}
\begin{karte}{Definition Kanonische Basis}
    \(e_i \in K^n\), \(i \in \set{1, \ldots, n}\) ist der 
    \(i\)-te Basisvektor der kanonischen Basis
    \[ \set{e_1, \ldots,e_n}, \] für den gilt, dass die 
    \(i\)-te Komponente \(1\) ist und alle andere 
    Komponenten \(0\) sind.
    \[ \left(
        \begin{array}{c}
        \lambda_1 \\
        \vdots \\
        \lambda_n
        \end{array}
        \right) = \lambda_1 e_1 + \ldots + \lambda_n e_n. \]
\end{karte}
\begin{karte}{Erweiterungslemma}
    Sei \(T \subset V\) linear unabhängig und \(v \in V \backslash T \). Dann gilt: 
    \[T \cup \set{v} \text{ linear unabhängig } 
    \Leftrightarrow v \notin \langle T \rangle. \]
\end{karte}
\subsection*{Dimensionen}
\begin{karte}{Basisergänzungssatz}
    Sei \(S \subset V\) ein Erzeugendensystem und \(T \subset V\)
    linear unabhängig. \\
    Dann kann man \(T\) durch Hinzunahme geeigneter Vektoren aus \(S\)
    zu einer Basis ergänzen.
\end{karte}
\begin{karte}{Lemma von Zorn}
    Ist \(\mathcal{M}\) eine teilweise geordnete Menge und besitzt
    jede streng geordnete Teilmenge von \(\mathcal{M}\) eine obere 
    Schranke in \(\mathcal{M}\), so besitzt \(\mathcal{M}\) 
    ein maximales Element.
\end{karte}
\begin{karte}{Austauschlemma}
    Seien \(B, B'\) zwei Basen eines Vektorraums \(V\). \\ 
    Für jedes \(v \in B \) gibt es ein \(w \in B'\) so, 
    dass \((B \setminus \set{v}) \cup \set{w}\) 
    eine Basis von \(V\) ist.
\end{karte}
\begin{karte}{Eigenschaften einer Basis}
    Für \( B \subset V \) sind äquivalent: 
    \begin{itemize}
        \item \(B\) ist eine Basis von \(V\).
        \item \(B\) ist eine maximal linear unabhängige Menge.
        \item \(B\) ist ein minimales Erzeugendensystem.
    \end{itemize}
\end{karte}
\begin{karte}{Definition Dimension}
    Die Mächtigkeit einer Basis von \(V\) heißt Dimension von \(V\). \\ 
    Je zwei Basen eines Vektorraums \(V\) haben die gleiche Mächtigkeit. \\
    Notation: \( \dim V \) oder \( \dim_K V \).
\end{karte}
\subsection*{Konstruktion von Vektorräumen}
\begin{karte}{Summe von Vektorräumen}
    Für \(U_1, U_2 \subset V\) Untervektorräume ist 
    \[U_1 + U_2 := \set{x + y \;\vert \; x \in U_1, y \in U_2}\]
    die Summe von \(U_1\) und \(U_2\).
\end{karte}
\begin{karte}{Definition Direkte Summe von Vektorräumen}
    Die direkte Summe \(U = U_1 \oplus U_2\) von zwei 
    Untervektorräumen \(U_1\) und \(U_2\) zeichnet sich durch 
    folgende Äquivalenzen aus: 
    \begin{itemize}
        \item Jedes \(v \in U\) lässt sich schreiben als 
        \(v = v_1 + v_2\) mit eindeutig bestimmten Vektoren
        \(v_i \in U_i, i \in \set{1,2}\).
        \item \(U_1 \cap U_2 = \set{0}\).
    \end{itemize}
\end{karte}
\begin{karte}{Definition Komplement}
    Seien \(U,W \subset V\) Untervektorräume.\\
    Gilt \(V = U \oplus W\), so ist W das Komplement von \(U\). \\
    Jeder Untervektorraum besitzt ein Komplement.\\
    Ist \( V = U \oplus W \), so gilt \( \dim V = \dim U + \dim W \).
\end{karte}
\begin{karte}{Definition Äußere Direkte Summe}
    Die \textit{äußere direkte Summe} der \(K\)-Vektorräume \(V\) 
    und \(W\) ist der\\
    \(K\)-Vektorraum gegeben durch \( V \times W \) 
    mit der Addition und der skalaren Multiplikation 
    \[ (x,y) + (x',y') := (x + x', y + y'), \quad 
    \alpha \cdot (x,y) := (\alpha \cdot x, \alpha \cdot y) \]
    für \( x,x' \in V, y,y' \in W \) und \( \alpha \in K \). 
    Man notiert sie mit \( V \oplus W \).
\end{karte}
\begin{karte}{Satz der Untervektorraumdimension}
    Sei \( U \subset V \) ein Untervektorraum und \( \dim V < \infty \). \\
    Wenn \( \dim U = \dim V \), so ist \( U = V \).
\end{karte}
\begin{karte}{Dimensionsformel für Untervektorräume}
    Seien \( U_1, U_2 \) Untervektorräume von \(V\). Dann gilt 
    \[ \dim(U_1 + U_2) = \dim U_1 + \dim U_2 - \dim(U_1 \cap U_2). \]
\end{karte}
\begin{karte}{Quotientenvektorraum}
    \( V / U \) sei die Quotientenmenge \( V /_{\sim U} \). Dann sind durch 
    \begin{align*}
        [x] + [y] := [x + y]\\
        \lambda [x] := [\lambda x]
    \end{align*}
    für alle \( x,y\in V, \lambda \in K \) eine Addition und skalare Multiplikation 
    für \( V/U \) definiert.
\end{karte}
\begin{karte}{Dimensionsformel für Quotientenvektorräume}
    Sei \( U \subset V \) ein Untervektorraum. Dann gilt 
    \[ \dim U + \dim(V/U) = \dim V. \]
\end{karte}
\end{document}
