\documentclass[main.tex]{subfiles}
\begin{document}

\section*{Determinaten}

\subsection*{Die Determinantenfunktion}

\begin{karte}{Definition \(n\)-Linearität, alternierend und Antisymmetrie}
    Eine Abbildung \(f: M_n(K) \rightarrow K\) heißt
    \begin{itemize}
        \item \textit{\(n\)-linear}, wenn sie in jeder Zeile linear ist. \( (*) \)
        \item \textit{alternierend}, wenn gilt: \\
        Hat \(A \in M_n(K)\) zwei gleiche Zeilen, so ist  \(f(A) = 0\).
       \item \textit{antisymmetrisch}, wenn gilt:\\
        Entsteht \(A'\) aus \(A \in M_n(K)\) durch vertauschen zweier
        Zeilen, so ist \(f(A') = -f(A)\).
    \end{itemize}
    \( (*) \) \textit{\(n\)-linear} heißt: \\
    Für jedes \(j \in \set{1,\ldots,n}\) und \(\forall v_1, 
    \ldots,v_{j-1},v_{j+1},\ldots,v_n \in M_{1,n}(K)\) ist 
    die \\ 
    folgende Abbildung linear:
    \[K^n \rightarrow K, v \mapsto f(v_1, \ldots,v_{j-1},v,v_{j+1},\ldots,v_n).\]
\end{karte}
\begin{karte}{Eigenschaften \(n\)-linearer, alternierender und antisymmetrischer Abbildungen}
    \begin{itemize}
        \item Eine \(n\)-lineare und alternierende Abbildung 
        \(f: M_n(K) \rightarrow K\) ist antisymmetrisch.
        \item Sei \(f: M_n(K) \rightarrow K\) \(n\)-linear und alternierend.
        Elementare Zeilenumformungen verändern den Wert von \(f\) wie folgt: 
        \[f(V_{j,k} \cdot A) = -f(A)\]
        \[f(S_{j,k}(\lambda) \cdot A) = f(A)\]
        \[f(M_j(\lambda) \cdot A) = \lambda \cdot f(A)\]
    \end{itemize}
\end{karte}
\begin{karte}{Definition Determinate}
    Für jedes \(c \in K\) gibt es genau eine \(n\)-lineare alternierende
    Abbildung \(\Delta_c: M_n(K) \rightarrow K\) mit \(\Delta_c(I_n) = c \).
    Die \textit{Determinante} ist definiert als: 
    \[\det := \Delta_1: M_n(K) \rightarrow K.\]
\end{karte}
\begin{karte}{Entwicklung der Determinante nach der \(j\)-ten Spalte}
    Für \(A = (a_{pq}) \in M_n(K)\) und \(j \in \set{1, \ldots ,n}\) gilt:
    \[\det A = \sum_{i=1}^n {(-1)}^{i+j}a_{ij} \cdot \det A[i,j].\]
    Beispiel: \\
    Wir entwickeln eine \(2 \times 2\)-Matrix \(A = (a_{ij})\) nach der
    ersten Spalte: 
    \[ \begin{split}
        \det 
        \begin{pmatrix}
        a_{11} & a_{12} \\
        a_{21} & a_{22}    
         \end{pmatrix} 
    & = {(-1)}^{1+1}a_{11} \cdot \det [1,1] + (-1)^{2+1}a_{21} \cdot \det A[2,1] \\
    & = a_{11} \cdot a_{22} - a_{21} \cdot a_{12}\\
    & = \text{Produkt Diagonale} - \text{Produkt Antidiagonale}
    \end{split} 
    \]
\end{karte}
\begin{karte}{Determinante einer oberen Dreiecksmatrix} 
    Ist \(A = (a_{ij}) \in M_n(K)\) eine obere Dreiecksmatrix, d.\ h.\ 
    \(a_{ij} = 0\) für \(i > j\), dann ist \(\det A\) das Produkt der 
    Diagonalelemente: 
    \[\det A = a_{11} \cdots a_{nn}\]
    Bei großen Matrizen ist die Entwicklung nach Spalten meist keine 
    gute Berechnungsmethode. Stattdessen: 
    \begin{itemize}
       \item Man überführt \(A\) in  eine obere Dreiecksmatrix 
       \(A'\) durch Zeilenvertauschungen und Zeilenumformungen des 
       Typs Scherung.
       \item Ist \(r\) die benötigte Anzahl von Zeilenvertauschungen, so gilt:
       \[\det A = {(-1)}^r \cdot \det A' = {(-1)}^r \cdot a'_{11} \cdots a'_{nn} .\]
    \end{itemize}
\end{karte}

\subsection*{Determinante und Matrixprodukt}

\begin{karte}{Determinanten von Matrixprodukten und transponierten Matrizen}
    \begin{itemize}
        \item \(\forall A,b \in M_n(K) \text{ gilt } 
        \det(A \cdot B) = \det A \cdot \det B\).
        \item Für jedes \(A \in M_n(K) \text{ gilt } 
        \det A^T = \det A\).
    \end{itemize}
\end{karte}
\begin{karte}{Definition komplementäre Matrix}
    Für \(A \in M_n(K)\) ist \(\tilde{A} = (\tilde{a}_{ij}) \in M_n(K)\) mit
    \[\tilde{a}_{ij} := {(-1)}^{i+j} \cdot \det A[j,i]\]
    die zu \(A\) \textit{komplementäre Matrix}.
    Des Weiteren gilt:
    \begin{itemize}
        \item \(\tilde{A} \cdot A = (\det A) \cdot I_n\)
        \item \(A\) ist genau dann invertierbar, wenn \(\det A \neq 0\). Dann gelten: 
        \[\det(\inverse A) = \inverse{(\det A)} \text{ und }
         \inverse A = \inverse{(\det A)} \tilde{A}.\]
    \end{itemize}
\end{karte}

\subsection*{Determinante von Endomorphismen}

\begin{karte}{Definition Endomorphismus\\und dessen Determinante}
    Ein \textit{Endomorphismus} ist eine lineare Abbildung 
    \( f: V \rightarrow V \).\\
    Die Determinante eines Endomorphismus über einen endlich-dimensionalen 
    Vektorraum \(V\) ist definiert als:
    \[ \det f := \det M_{BB}(f) \in K \]
    für eine Basis \(B\) von \(V\).\\
    \(\det M_{BB}(f)\) hängt nicht von der Wahl von B ab. 
\end{karte}
\begin{karte}{Folgerungen aus der Determinante eines Endomorphismus}
    Sei \(f: V \rightarrow V\) ein Endomorphismus und \(\dim(V) < \infty\).
    \begin{itemize}
        \item \(\det f \neq 0 \Leftrightarrow f\) ist Isomorphismus.
        \item \(\det g \circ f = \det g \cdot \det f\).
        \item \(\det \id_V = 1\).
    \end{itemize}
\end{karte}

\subsection*{Die Leibnizformel}

\begin{karte}{Leibnizformel}
    Für jedes \(A = (a_{ij}) \in M_n(K)\) gilt:
    \[\det A = \sum_{\sigma\in\Sigma_n} 
    \text{sign}(\sigma)\cdot a_{1\sigma(1)} \cdots a_{n\sigma(n)}. \]
\end{karte}
\begin{karte}{Defintion Permutation}
    Eine Permutation \(\sigma \in \Sigma_n\) ist eine Abbildung, die 
    Elemente von \((1, \ldots, n)\) vertauscht.\\
    \(\Sigma_n\) ist wie folgt definiert: 
    \[\Sigma_n := \set{f: \set{1, \ldots ,n} \rightarrow \set{1, \ldots ,n} 
    \;\vert \; f \bijektiv}\]
    Des Weiteren ist: 
    \begin{itemize}
        \item Eine Permutation, die zwei Zahlen vertaucht und die 
        anderen festlässt, heißt \textit{Transposition}.
        \item Eine \textit{Transposition}, die benachbarte Zahlen 
        vertauscht, heißt \textit{Nachbarnvertauschung}.
        \item Eine Permutation, die Komposition einer geraden Anzahl 
        von Nachbarnvertauschungen ist, heißt \textit{gerade}, 
        ansonsten \textit{ungerade}.
    \end{itemize}
\end{karte}
\begin{karte}{Definition Signum einer Permutation}
    Für eine Permutation \(\sigma \) definiere man: 
    \[ \text{sign}(\sigma) := \begin{cases}
        1 & \text{falls } \sigma \text{ gerade} \\
        -1 & \text{falls } \sigma \text{ ungerade}
    \end{cases}. \]
\end{karte}
\begin{karte}{Definition Homomorphismus von Gruppen}
    Eine Abbildung \(f: G \rightarrow H\) zwischen Gruppen, die 
    \(\forall g_1,g_2 \in G\) folgende Bedingung erfüllen, \\
    heißen \textit{Homomorphismus (von Gruppen)} oder 
    \textit{Gruppenhomomorphismus}:
    \[f(g_1 \cdot g_2) = f(g_1) \cdot f(g_2).\]
    Es gilt: 
    \begin{itemize}
        \item \(\text{sign}: \Sigma_n \rightarrow \set{-1,1}\) 
        ist ein Homomorphismus.
        D.\ h.\ für alle Permutationen \(\sigma, \tau \) gilt:
        \[\mathrm{sign}(\sigma \circ \tau) = \mathrm{sign}(\sigma) 
        \cdot \mathrm{sign}(\tau).\]
        \item Das Signum einer Transposition ist \(-1\).
    \end{itemize}
\end{karte}

\end{document}